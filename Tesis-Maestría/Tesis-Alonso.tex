\documentclass[12pt]{report}

%
\setcounter{tocdepth}{3}
\setcounter{secnumdepth}{3}
\newcommand{\matr}[1]{\mathbf{#1}}
\usepackage[usenames]{color}
\newcommand{\angstrom}{\textup{\AA}}
\usepackage{lmodern}
%\usepackage[useTeX]{mmap} 
\usepackage[utf8]{inputenc}
\usepackage{nccmath}
\usepackage[T1]{fontenc}
\usepackage{physics}
\usepackage[spanish, es-nodecimaldot, es-tabla, es-nolayout]{babel}
\usepackage{array, booktabs, dcolumn,multirow}
\usepackage[margin=1.5 in]{geometry}
\usepackage{mathtools, amssymb}
\usepackage{mathrsfs}
\usepackage{microtype}
\usepackage{xcolor}
\usepackage{xfrac}
\usepackage{graphicx} 
\setlength{\parindent}{0cm}
\usepackage{enumerate}
\usepackage{multirow}
\usepackage{hyperref}
\usepackage{float}
\usepackage{subcaption}
\usepackage[superscript]{cite} %% stye in doc, normal in list of ref
\usepackage{makeidx}
%\makeindex
%
\usepackage{etoolbox}
\makeatletter
\patchcmd{\chapter}{\if@openright\cleardoublepage\else\clearpage\fi}{}{}{}
\makeatother
\addtolength{\textwidth}{3 cm}                                                  
\addtolength{\textheight}{2 cm}                                                 
\addtolength{\topmargin}{-1.5 cm}                                               
\addtolength{\oddsidemargin}{-1.5 cm}                                           
\addtolength{\evensidemargin}{-1.5 cm}                                          
\linespread{1.5} 
\title{"Estudio dinámico y estructural, de las especies doblemente protonadas XH$_4^{2+}$ y XH$_4^+$ (X=O,S,Se,Te,Po)"} 
\vspace{1cm}
\author{\large{Lic. Alonso Daniel Jacobo Hernández}}
\renewcommand*\contentsname{Índice general}
%
\usepackage{mathptmx}
\usepackage{titlesec}
\titleformat{\chapter}[block]
  {\normalfont\huge\bfseries}{\thechapter.}{1em}{\Huge}
\titlespacing*{\chapter}{20pt}{-19pt}{20pt}
\usepackage{transparent}
\usepackage{eso-pic}

\begin{document}

\AddToShipoutPicture*{
\put(0,0){
\parbox[b][\paperheight]{\paperwidth}{%
\vfill
\centering
{\transparent{0.1}
\includegraphics[scale=0.5]{figuras/1200px-Uaem_Morelos_logo.png} 
}%
\vfill
}
}
}
\vspace*{\fill}
El presente trabajo se realizó en el Centro de Investigaciones químicas de la Universidad Autónoma del Estado de Morelos (CIQ-UAEM), bajo la dirección del Dr. Ramón Hernández Lamoneda, y con el apoyo del Consejo Nacional de Ciencia y Tecnología (CONACyT)a través de la beca 867524.

\newpage

\tableofcontents
\newpage
\listoffigures
\newpage
\listoftables
\newpage
\addcontentsline{toc}{chapter}{Resumen}
\chapter*{Resumen}
Las especies tetraédricas H$_4$X${^{n+}}$ ($n=2, 1, 0$; X=calcógeno) fueron estudiadas a nivel CASSCF con el espacio activo de valencia completo, y la inestabilidad originada por el efecto de Jahn-Teller de segundo orden fue analizada para las especies H$_4$X${^{+}}$ y H$_4$X. Todas las especies H$_4$X$^{2+}$ son mínimos locales, pero todos los radicales H$_4$X$^+$ y las especies hipervalentes H$_4$X son inestables como lo muestra la curvatura negativa asociada a lo largo de los modos de estiramiento $T_2$. La inestabilidad del H$_4$O$^+$ fue comparada con la estabilidad del radical isoeléctrico NH$_4$. En paralelo con la argumentación del acoplamiento vibrónico, se construyo un modelo de particion de la energia total para analizar el origen de la curvatura negativa.
\\


Demostramos que el decrecimiento precipitado de la VAE es en respuesta a la curvatura negativa $T_2$ en el H$_4$O$^+$. La diferencia en la estabilidad de los radicales fue rastreada a las diferencias en la estructura electrónica mediante un analisis de densidad electronica y la carga contenida en un volumen especificado.

\newpage


\chapter{Introducción}
Las moléculas altamente protonadas son relevantes en el diseño de materiales para el almacenamiento de hidrogeno\cite{Karkamar2009}. La existencia del agua y ácido sulfhídrico diprotonados fue postulado por Olah et al.\cite{Olah1986, Olah1988}  basándose en un estudio de resonancia magnética nuclear de las reacciones de  intercambio de hidrogeno con deuterio en medios super ácidos. Los autores sugieren que el intercambio D/H ocurre a través de especies dicationicas como intermediarios de reacción, dichas especies fueron estudiadas posteriormente mediante un cálculos ab initio, con los cuales se demostró la estabilidad cinética y termodinámica del H$_4$O$^{2+}$ y H$_4$S$^{2+}$. Estos hallazgos fueron precedidos por cálculos de Kozmuta et al.\cite{Kozmuta1982} y Choi et al.\cite{Choi1988} seguidos y recalculados de una manera más precisa con un estudio de estructura electrónica a nivel (QCISD(T)/6-311G(2df,2p) ) por Boldyrev y Simons\cite{Boldyrev1992}. Este último estudio confirmaba la existencia del tetraedro H$_4$O$^{2+}$ como un mínimo local con una barrera de 38.1 kcal/mol separándolo de los productos H$_3$O$^+$ + H$^+$, los cuales son térmicamente favorecidos por 61.3 kcal/mol. Estos autores reportaban que al reducir el dication H$_4$X$^{2+}$ con un solo electrón se abre el camino a los productos H + H$_3$X$^+$. La energía liberada por la descomposición del monocatión H$_4$O$^+$ es referida como “energía de explosión” en dicho artículo y esta es de 65 Kcal/mol.
\\

La afinidad electrónica vertical de los dicationes H$_4$X$^{2+}$, es aproximadamente entre 13-14 eV, esta excede al primer potencial de ionización de la mayoría de las moléculas.  Esto provoca que la reducción del H$_4$X$^{2+}$ a H$_4$X$^+$ sea altamente probable en medios químicos comunes.  Centrándonos en el H$_4$O$^{2+}$ y el H$_4$O$^+$, en el H$_4$O$^{2+}$ la estabilidad electrónica única, similar a la especie isoelectrónicas Mg$^{2+}$, es aparentemente, capaz de contrarrestar la repulsión Coulombica y esto hace que la estructura electrónica tetraédrica sea localmente estable. ¿Por qué el H$_4$O$^+$ es instable? Basándonos en la analogía con el estado basal ([Mg$^{2+}$]3s (2S)) y el primer estado excitado ([Mg$^{2+}$]3p (2P)) los estados electrónicos del Mg$^+$ están separados por 4.4 eV. Por lo que nosotros esperamos que el estado basal $^2$A$_1$ y el estado excitado $^2$T$_2$ del H$_4$O$^+$ sean cercanos en energía y por lo tanto susceptibles al acoplamiento vibrónico. Nuestra hipótesis, probada en este trabajo, es que el acoplamiento vibrónico entre los estados $^2$A$_1$ y $^2$T$_2$  da origen a una curvatura negativa para un conjunto de modos vibracionales de simetría $T_2$ del H$_4$X$^+$, los cuales están dominadas por los estiramientos XH. La curvatura negativa provocaría la descomposición del H$_4$X$^+$ a H + H$_3$X$^+$ (X=O,S) sin barrera, como observaron Boldyrev and Simons. En este trabajo caracterizamos el origen de la inestabilidad debido al efecto de Jahn-Teller de segundo orden (por sus siglas en ingles SOJT) con el método de espacio activo completo de campo autoconsistente (por sus siglas en ingles CASSCF) utilizando el formalismo desarrollado por Bearpark, Blancafort y Rob\cite{Bearpark2002}. La estabilidad local del H$_4$X$^{2+}$ y la inestabilidad del H$_4$X$^+$ también será ilustrada para los demás calcógenos, Se, Te, y Po.  
\\


El H$_4$O$^{2+}$ no es isoeléctrico únicamente con Mg$^{2+}$, sino que también es isoeléctrico con el NH$_4^+$, aunque este último es equivalente al Na$^+$ y no al Mg$^{2+}$ debido a la carga total en cada especie. La estabilidad geométrica del radical tetraédrico neutro NH$_4$ ($^2$A$_1$) fue predicha por Herzberg\cite{Herzberg1984} el cual hizo una detección espectroscópica  de la especie y caracterizándola con el espectro de emisión completo de dicha especie, como si el acoplamiento entre los estados electrónicos $^2$A$_1$ y $^2$T$_2$ fuera mucho más débil que en el caso de los radicales H$_4$X$^+$ (X=calcógeno) discutidos anteriormente. Un hecho sorprendente a primera vista, es que la separación 2P – 2S en el átomo de Na$^+$ es de 2.2 eV comparada con la del Mg$^+$ de 4.4 eV. La estabilidad geométrica del radical $^2$A$_1$ del NH$_4$ condujo a la idea de un “enlace químico de Rydberg” propuesto por Boldyrev y Simons\cite{Boldyrev2002}. Estos resultados nos llevaron a analizar el acoplamiento vibrónico $^2$A$_1/^2$T$_2$ en el NH$_4$ al mismo nivel que el XH$_4^+$ (X=calcógeno) para aclarar las similitudes y diferencias entre estos sistemas.
\\


Por otra parte, la existencia del anión tetraedro NH$_4^-$ ( $^1$A$_1$) fue postulado por el grupo de Nilles\cite{Nilles2002} basado en su información obtenida mediante espectroscopia foto electrónica. Las propiedades espectroscópicas del NH$_4^-$ sugieren una gran similitud estructural entre el NH$_4^-$ y el NH$_4$, una conclusión confirmada computacionalmente por diversos grupos. Claramente, tiene que existir un acoplamiento vibrónico entre los estados $^1$A$_1$ y $^1$T$_2$ en el NH$_4^-$, pero su fuerza es lo suficientemente débil para hacer que el NH$_4^-$ sea geométricamente estable.


\newpage
\section{Recciones en medios súper ácidos}

La primera aparicion del dicatión H$_4$O$^{2+}$ fue en una reacción propuesta Olah et al\cite{Olah1986}. En la cual se estudiaba el intercambio del ion deuterio por protones para el ion hidronio en medios super ácidos, caracterizado por RMN. 

\begin{figure}[h]
\centering
\includegraphics[trim={1cm 9.5cm 4cm 3cm},clip,scale=0.6]{figuras/reaccionolah.pdf} 
\caption{Mecanismos de reacción propuestos para el intercambio hidrógeno-deuterio en medios super ácidos.\cite{Olah1986}}
\end{figure}

Para esta reacción se propusieron dos mecanismo posibles.El primero tenía como característica principal que el primer paso era el de perder un protón, contrario al segundo en el cual el primer paso era ganar un protón, este además tiene la característica de formar el H$_4$O$^{2+}$ como intermediario de reacción. Posteriormente se repitió la reacción en un medio aún más acido, teniendo como resultado esta vez que la reacción era más rápida, con esta información experimental pudieron concluir que el mecanismo que reproducía el comportamiento experimental era el segundo, esto debido al primer paso que era el de ganar un protón, el cual sería más rápido si el medio era aún más acido. Con esta información se realizó un estudio computacional en el cual buscaba confirmar que el H$_4$O$^{2+}$ era estable, confirmando que la especie era un mínimo geométrico en la superficie de energía potencial (SEP) en la forma de un tetraedro.\\

\newpage
Posterior a este estudio, se realizó uno muy similar \cite{Olah1988} pero esta vez estudiando el intercambio de H/D para el ion sulfonio, obteniendo resultados experimentales equivalentes que en el ion hidronio, una vez más se realizaron cálculos para confirmar la estabilidad geométrica del dication, dando como resultado que la especie era aún más estable que la del oxígeno, esta conclusión tomada a partir de cálculos sobre la estabilidad termodinámica de las especies.\\

\begin{figure}[h]
\centering
\includegraphics[trim={5cm 15cm 4cm 2cm},clip,scale=0.6]{figuras/reaccionolah2.pdf} 
\caption{Proceso termodinámico para la obtención del H$_4$S$^{2+}$ a partir del H$_4$O$^{2+}$.\cite{Olah1988}}
\end{figure}


Es importante recordar que los cálculos son en condiciones aisladas en donde el SH$_3^+$ es termodinámicamente desfavorable. Sin embargo, los efectos de solvatación juegan un rol importante en la fase condensada donde es posible obtener un superácidos bien estructurado a partir del SH$_3^+$, enlazandoce entre ellos mediante puentes de hidrogeno.

\begin{figure}[h]
\centering
\includegraphics[trim={1cm 1cm 1cm 1cm},clip,scale=0.19]{figuras/reaccionolah3.pdf} 
\caption{Estructura de un super acido formado a partir del SH$_3^+$ en fase condensada, que contine especies similares a los dicationes con simetría $C_{3v}$.\cite{Olah1988}}
\end{figure}

\newpage
Estos dos artículos sirvieron como precedentes sobre la posible existencia de los dicationes H$_4$O$^{2+}$ y H$_4$S$^{2+}$ en medio super ácidos, recientemente surgió un estudio de Stoyanov et al.\cite{Stoyanov2012} en donde investigan una serie de especies con un oxigeno tetra coordinado de la forma R$_3$O$^+\dotsi$H$^+$. Esta especie presenta muchas similitudes a la propuesta por Olah et al. En este articulo ellos mencionan que se puede generar especies de la forma R$_3$OH$^{2+}$ mediante la alquilación aromática con sales de R$_3$O$^+$. Una de estas especies es el Me$_3$OH$^{2+}$ el cual resultó ser más estable que el Me$_3$O$^+$ por 27.6 kcal/mol. Sin embargo, las reacciones de intercambio o alquilación en las que pueden generar este tipo de intermediarios no se dan en fase gas, sino más bien en fase liquida en medios súper ácidos los cuales pueden cambiar el panorama dramáticamente. Por ejemplo, el H$_4$O$^{2+}$ no es un mínimo en algunos campos de reacción en modelos de solvente continuo. Para el caso del Me$_3$OH$^{2+}$ hay evidencia de que no hay una transferencia de protón completa en esta especie, si no que la activación se da a través de puentes de hidrogeno sobre el oxígeno tetravalente, Me$_3$O$^+\dotsi$ H$^+$.
\\


Una de las propuestas más prometedoras de las especies R$_3$O$^+\dotsi$H$^+$ fue hecha por Schneider y Werz\cite{Schneider2010} en la cual los sustituyentes R eran policíclicos, esto para ayudar a estabilizar los calcógenos utilizados como centro tetra coordinado, esta propuesta era innovadora ya que se esperaba que estas especies fueran observables en un medio inerte. La especie investigada que mejores resultados obtuvo fue el ion básico oxatriquinano el cual es un ion que cuenta con un oxigeno triplemente coordinado con sustituyentes pentacíclicos. Este ion fue estabilizado mediante el super acido de Br\o nsted H(CHB$_{11}$Cl$_{11}$). Esta combinación de especies resulto ser estable en solución uniéndose mediante un enlace de hidrogeno\cite{Stoyanov2012}. 


\begin{figure}[h]
\centering
\includegraphics[trim={1cm 1cm 1cm 1cm},clip,scale=0.24]{figuras/Stoyanov.pdf} 
\caption{Estructura de equilibrio formada entre el oxatriquinano y el super acido (CHB$_{11}$Cl$_{11}$) unidos mediante un puente de hidrógeno.\cite{Stoyanov2012}}
\end{figure}

\newpage
\section{Almacenamiento de hidrogeno }

El hidrógeno como combustible es atractivo debido a su alta eficiencia energética y limpieza al ser quemado, sin embargo, su baja densidad hace que su utilización en forma gaseosa no sea práctica para ciertas aplicaciones, por lo que es de gran interés encontrar formas alternativas para su almacenamiento y uso. Debido a esta problemática surge la idea de almacenar hidrógeno en especies químicas. Un ejemplo de estas moléculas es el borohidruro de amonio: [NH$_4$]$^+$[BH$_4$]$^-$\cite{Parry2002} el cual fue generado experimentalmente mediante una reacción de metátesis, la cual consiste en un intercambio de enlaces entre dos especies químicas que reaccionan, como se puede ver en la figura \ref{metatesis}.  El borohidruro de amonio puede liberar hasta un 21\% de hidrógeno por peso a temperaturas menores a 160    $^\circ$C \cite{Karkamar2009}, mediante una reacción en varios pasos. Esto fue confirmado a partir espectrometría de masas y la utilización de una bureta de gases para contener el hidrógeno liberado. 


\begin{figure}[h]
\centering
\includegraphics[trim={5cm 16cm 5cm 2cm},clip,scale=0.7]{figuras/Parry.pdf} 
\caption{Reacción de metátesis para generar la especie [NH$_4$]$^+$[BH$_4$]$^-$ donde M es el ion amonio para nuestro caso de interes.\cite{Parry2002}}
\label{metatesis}
\end{figure}


\begin{table}[h]
\centering
\begin{tabular}{lllll}
\hline
Paso & Material de inicio & Productos & $T_d(^\circ C)$ & $\Delta H $ (kJ/mol)  \\ \hline
1a &[NH$_4$]$^+$[BH$_4$]$^-$  & NH$_3$BH$_3$+H$_2$ &  &  \\
1b &\begin{tabular}[c]{@{}l@{}} [NH$_4$]$^+$[BH$_4$]$^-$\\ +NH$_3$BH$_3$\end{tabular}  &  \begin{tabular}[c]{@{}l@{}} [NH$_3$BH$_2$NH$_3$]$^+$[BH$_4$]$^-$\\ +H$_2$\end{tabular} & 50 & -40 \\
2 &[NH$_3$BH$_2$NH$_3$]$^+$[BH$_4$]$^-$  &  PAB + H$_2$&85  &-15  \\
3 &  PAB + H$_2$ &  PIB + H$_2$ & 130 & -13 \\ \hline
\end{tabular}
\caption{Canal de reacción, temperatura de descomposición $T_d$ y entalpia de reacción asociada al liberar hidrógeno, para la especie [NH$_4$]$^+$[BH$_4$]$^-$ .\cite{Karkamar2009}}
\end{table}

\newpage

La idea principal por la cual se plantea usar moléculas cargadas es que estas pueden ser estabilizadas por algún contraión en forma sólida, además de que los enlaces en estos iones generalmente son más débiles que los enlaces de una molécula neutra.  En este sentido es natural considerar otras especies con un alto contenido en hidrógenos como H$_3$O$^+$, esta última podría ser transformada en la especie H$_4$O$^{2+}$   gracias al par de electrones libre que tiene el oxígeno.
\\



Como se discutió en la sección anterior las especies 	H$_4$X$^{2+}$	y R$_3$X$^+\dotsi$H$^+$ pueden ser obtenidas en medios super ácidos, 	por lo que la idea de estabilizar estas especies con contraiones para formar sales estables es factible y merece ser investigada. Sin embargo, hay que estudiar las reacciones como acido de Lewis o Bronsted y buscar mecanismos que permitan estabilizar al sistema completo.


\newpage

\chapter{Hipótesis}

Los dicationes de la forma: XH$_4^{2+}$, son de sumo interés debido a que se plantean como moléculas capaces de almacenar hidrógeno, el cual es importante debido a su gran aplicación como combustible limpio. Estas, sin embargo, al ser dicationes, se comportan como un super acido, tanto bajo la definición de ácido de Lewis como de Br\o nsted-Lowry, en trabajos anteriores\cite{Alonso} se demostró que el H$_4$O$^{2+}$ y H$_4$S$^{2+}$ son geométricamente estables, y a su vez cinéticamente estables frente a la transferencia de protón a algunas especies, sin embargo, para la transferencia de electrón, en el caso del H$_4$O$^+$ esta especie resulto ser inestable debido a la presencia del efecto de Jahn-Teller de segundo orden.
\\


Para el caso del H$_4$S$^+$ y los demás calcógenos el problema es difíci, esto debido a que los resultados cambiar según el método y la base que utilizamos, esto hace que sea complicado sacar una conclusión sobre la estabilidad geométrica de los monocationes. Para el caso específico del H$_4$S$^+$ el cual fue el más estudiado con anterioridad, podemos ver que los valores de las frecuencias de estiramiento T$_2$ varían mucho, desde valores irrealmente grandes a nivel MP2, hasta discrepancias entre si las frecuencias son o no reales.
\\




\begin{table}[H]
\centering
\begin{tabular}{l|lllllll}
\hline
 & HF & MP2 & CCSD & CASSCF &RS2C & B3LYP & PBE   \\ \hline
$T_2$ & 1264.4 & 1086.0  & 276.0 & 978.4 &977.60  &1109.8 &  1116.9 \\
$E$  &  763.5 & 559.8 & 538.3  & 500.0 & 590.9 & 512.0 &481.6  \\
$T_2$  & 8011.5 & 10306.5 &1857.6  & 4765.9 $i$  & 4569.03 & 2785.3 $i$ &  2271.3 $i$\\
$A_1$  & 2046.1 & 2013.9 & 1878.2 & 1721.4 &1813.34 &1861.1  & 1858.1\\ \hline
\end{tabular}
\caption{Frecuencias harmónicas vibracionales para el H$_4$S$^+$ con diferentes métodos utilizando la base aug-cc-pVDZ.}
\label{hipotesistabla}
\end{table}

\newpage

La hipótesis sobre el origen del problema es que el estado basal $^2$A$_1$ y los estados excitados $^2$T$_2$ son muy cercanos en energía, y a su vez tienen una interacción vibronica muy alta, por lo que el tratarlos por separado con métodos optimicen variacionalmente un estado
, no se describe de manera balanceada el problema y eso da lugar a errores que terminan por provocar los valores tan altos de las frecuencias, y en algunos casos el error en la interpretación sobre si son o no reales. Otro posible origen del problema es que muchos de los cálculos reportados en la tabla \ref{hipotesistabla} son obtenidos mediante derivadas numéricas 
las cuales son obtenidas con base en evaluaciones de la energia en geometrias que no tienen simetria y que por el problema de no balancear tus estados dan resultados espurios o incorrectos. Por lo que en este trabajo se busca esclarecer el problema al calcular frecuencias y en el caso de encontrar una inestabilidad en las especies, dar el origen de esta misma.
\\
\\
\\

\chapter{Objetivo }

En este trabajo nos enfocaremos en determinar si las especies H$_4$X$^{n+}$ (n=2, 1, 0; X=calcógeno) son estables, y en caso de no serlo determinar el origen de la inestabilidad, esto mediante cálculos de un solo estado (SS) y cálculos de promedio de estados (SA), con metodos de una sola referencia y multireferenciales. A su vez se comparará la inestabilidad de los monocationes H$_4$X$^+$ con especies isoelectrónicas, para tratar de dar una explicación a la inestabilidad de lo monocationes basándonos en un análisis de la naturaleza del orbital SOMO (singly occupied molecular orbital)
de dichas especies.

\newpage
\chapter{Fundamentación Teórica }
\section{Antecedentes}

Este estudio continua directamente el trabajo realizado por los grupos de los profesores Maciej Gutowski y Ramon Hernandez\cite{Ramon}, en el cual se estudiaban las estabilidades relativas de las especies H$_4$X$^{2+}$(X=O,S,Se,Te,Po) frente a diferentes procesos disociativos. En este trabajo se encontró que todas las especies eran termodinámicamente inestables ante la pérdida de un protón. Sin embargo, las especies resultaron ser cinéticamente estables al presentar barreras considerables frente dicho proceso, de entre 40 a 70 kcal/mol para los diferentes calcógenos. En ese mismo trabajo siguiendo la línea dejada por Boldyrev y Simons\cite{Boldyrev1992} se trató de estudiar la estabilidad de los dicationes con respecto a la transferencia de electrón. En ambos estudios se llegó a la conclusión de que para el caso X=O  es inestable ante la transferencia de protón ya que el monocatión H$_4$O$^+$ es inestable debido al efecto de Jahn-Teller de segundo orden. 
\\


Para el caso de los demás monocationes el estudio tomo un grado más de complejidad ya que dependiendo del cálculo y en algunos casos la base cambian los resultados sobre si la especie era estable, o en algunos casos los valores de las frecuencias no corresponden a lo esperado en terminos fisicos . En este sentido, es natural pensar que el próximo paso en la investigación era el de estudiar los monocationes a mayor detalle además de identificar el origen de la posible inestabilidad de estos.

\newpage
\section{Marco teórico}

\subsection{Métodos multiconfiguracionales}

En los métodos monoconfiguracionales en la teoría de Hartree-Fock (HF) y teoría de los funcionales de la densidad (DFT), describimos la función de onda con un solo determinante de Slater. Los multiconfiguracionales por otro lado, se construyen como una combinación lineal de varios determinantes, o configuraciones de estado (CSFs), cada CSF es una combinación lineal de determinantes adaptados por spin. Las funciones de onda multiconfiguracionales también tienen por nombre funciones de onda de interacción de configuraciones (CI).
\\


Existe un caso especial en el método de interacción de configuraciones en el cual todas los determinantes de Slater (o CSFs) de una simetría adecuada se incluyen en el proceso variacional. Por ejemplo, todos los determinantes de Slater obtenidos al excitar todos los electrones a todos los orbitales desocupados posibles. Este es conocido como cálculo de CI completo (FCI, por sus siglas en ingles), el cual provee la solución numérica exacta (utilizando una base infinita) a la ecuación de Schrödinger para un problema multiconfiguracional.



\subsection{CASSCF}
El método CASSCF es un método multiconfiguracional que se basa en la partición de los orbitales moleculares en subconjuntos, correspondientes a como se van a utilizar para construir la función de onda. Es necesario definir para cada bloque de simetría los siguientes subconjuntos:

\begin{enumerate}
\item Orbitales inactivos
\item Orbitales activos
\item Orbitales virtuales
\end{enumerate}


Los orbitales activos e inactivos están ocupados en la función de onda, mientras que los orbitales virtuales (también llamados secundarios o externos) no están ocupados y abarcan el resto del espacio orbital, definidos del conjunto de base utilizado para construir los orbitales moleculares. Los orbitales inactivos se mantienen doblemente ocupados en todas las configuraciones que son utilizadas para construir la función de onda CASSCF. El número de electrones ocupando estos orbitales debe ser el doble de los orbitales inactivos. Los electrones restantes (llamados electrones activos) ocupan los orbitales activos. 
%\newpage
\\



El método CASSCF es un intento de tratar de generalizar el modelo de Hartree–Fock para situaciones donde la quasi degeneración de estados electrónicos ocurre, ya que son sistemas donde la correlación estática cobra mucha importancia, mientras que intenta mantener la simplicidad de la aproximación RHF como sea posible. Técnicamente, el modelo CASSCF es por necesidad más complejo, desde el hecho de que está basado en una función de onda multiconfiguracional y acercarse a un cálculo FCI en espacio restringido. Los bloques de construcción son, como en el modelo RHF. Los orbitales ocupados (activos e inactivos). El número de electrones es, sin embargo, en general menos que el doble del número de orbitales.
\\


El número de configuraciones electrónicas es generado por el espacio orbital. La función de onda total es formada por una combinación lineal de todas las configuraciones, generadas a partir de realizar todas las posibles excitaciones de los electrones activos en los orbitales activos, que cumplen con los requerimientos del espacio dado y la simetría, además de tener los orbitales inactivos doblemente ocupados.  Los orbitales inactivos están presentes en todas las configuraciones, y son optimizados, estos orbitales tienen una ocupación exacta de dos, mientras que los números de ocupación de los orbitales activos es de entre cero y dos. Los orbitales inactivos deben de elegirse como los orbitales que se espera que no contribuyan a los efectos de correlación en la quasi degeneración.
\\



La simplicidad conceptual del modelo CASSCF caen en el hecho de que una vez que los orbitales activos e inactivos son elegidos, la función de onda está completamente especificada. En adición, tal procedimiento conduce a ciertas implicaciones computacionales en el proceso utilizado para obtener los orbitales ocupados y los coeficientes CI. El mayor desafío técnico inherente al método CASSCF es el tamaño de la expansión CI, $N_{CAS}$. Este está dado por la llamada formula de Weyl-Robinson\cite{Pauncz2018}




\begin{align}
N_{CAS}=\dfrac{2S+1}{n+1}\left(
\begin{matrix}
n+1  \\
\dfrac{N}{2}-S \\
\end{matrix}
\right)\left(
\begin{matrix}
n+1  \\
\dfrac{N}{2}+S+1 \\
\end{matrix}
\right)
\end{align}


Donde n es el número de orbitales activos, N el número de electrones activos, y $S$ el spin total $S$. $N_{CAS}$ incrementa fuertemente como función del tamaño de n de un espacio orbital activo. En la práctica, esto significa que hay un límite bastante estricto con el tamaño de este espacio. La experiencia muestra que ese límite se alcanza normalmente para n alrededor de 12-16 orbitales, excepto para casos con unos pocos electrones activos. El gran número de aplicaciones de CASSCF realizadas han mostrado hoy en día claramente que esta limitación no crea un problema grave normalmente. Debe recordarse que el modelo CASSCF es una extensión del esquema RHF. Como tal, se supone que produce buenas aproximaciones a la función de onda de orden cero cuando aparecen quasi degeneraciones. El método de CASSCF no fue desarrollado para tratar ningún efecto de correlación dinámica, pero provee un buen punto de partida para muchos estudios.
\\


Sin embargo, existen casos donde puede resultar ventajoso ser capaz de utilizar conjuntos más grandes de orbitales activos. Las dimensiones de una función de onda tipo CAS pueden volverse prohibitivamente grande, y puede ser de interés buscar otros medios para restringir el tamaño del desarrollo. Una posibilidad es modificar la partición CAS del espacio orbital a una forma más restringida (método RASSCF), esto puede mantener muchos más orbitales activos. Esto es hecho mediante la introducción de nuevas particiones orbitales, RAS1 y RAS3. RAS1 es parte del espacio inactivo, pero aquí se permiten uno o más electrones despareados. Los electrones pueden ser excitados de otros subespacios, RAS3 es parte del espacio virtual y es permitido que pueda tener uno o más electrones, esto es que, los electrones pueden ser excitados a este espacio y ahora, el espacio activo original es llamado RAS2. Tal función de onda es ilustrada en figura \ref{casras}, y muestra el caso con dos electrones en RAS2 acoplados a un triplete, dos electrones desapareados en RAS1 y dos en RAS3.


\begin{figure}[h]
\centering
\includegraphics[trim={1cm 1cm 1cm 1cm},clip,scale=0.5]{figuras/CAS.pdf} 
\caption{Ilustración de los orbitales activos para un CAS con 2 electrones y 3 orbitales activos, extendido a RAS SD.}
\label{casras}
\end{figure}

\newpage

\subsection{Promedio de estados MCSCF}

Una de las aplicaciones más comunes de los métodos CASSCF y RASCF es estudiar estados excitados para espectroscopia electrónica, procesos fotoquímicos y fotofísicos. Una aproximación, que usualmente funciona bien para tratar estos sistemas, es realizar cálculos de promedio de estados, donde los orbitales son optimizados para un número de estados electrónicos. La energía está escrita como un promedio de las energías de cada estado.

\begin{align}
E^{av}=N^{-1}\displaystyle\sum_{i}^{N} E_{i^{\prime}}
\end{align}

Donde $N$ es el número de estados considerados. 
Los cálculos de promedios de estados son necesarios en casos donde varios estados son cercanos en energía o donde el objetivo de la investigación, entre otras razones, involucra varios de estos como en espectroscopia, fotoquímica y fotofísica. Estos casos típicamente incluyen cruces evitados o intersecciones cónicas a lo largo de la coordenada de reacción. Otro caso típico es en espectroscopia electrónica donde estados excitados y estados de Rydberg son cercanos en energía.

\newpage
\subsection{SRCI}
Para cualquier sistema, el tratamiento más preciso posible es un cálculo del tipo FCI, el cual implica resolver el problema de valores propios utilizando todas las CSFs que pueden ser construidas utilizando el conjunto de base escogido.
\\


Asumiendo  $n_0$ orbitales espaciales, $n_\alpha$ y $n_\beta$ electrones con spin arriba y spin abajo, respectivamente, y sin un grupo de simetría, el número de determinantes de Slater está dado por\cite{Roos2016}

\begin{align}
N_{SD}=\left(
\begin{matrix}
n_0  \\
n_\alpha \\
\end{matrix}
\right)\left(
\begin{matrix}
n_0  \\
n_\beta \\
\end{matrix}
\right)
\end{align}

Usualmente, un número de orbitales de core pueden y deben estar congelados, esto es, dejarlos sin correlacionar y siempre doblemente ocupados. Dichos orbitales pueden ser optimizados mediante CASSCF, y utilizados sin alguna otra optimización.
\\


Una forma común de trabajar con cálculos de tamaño razonable o sistemas largos es truncando la expansión CI. Partiendo de un único determinante y donde los términos, “simples”, “dobles”, etc. se refieren a otros determinantes de Slater formados por reemplazar uno, dos, etc. spin-orbitales ocupados con orbitales virtuales que estaban desocupados en el determinante de referencia, es decir un CI mono referenciado, SRCI. Este es un esquema simple que puede funcionar muy bien. Los determinantes resultantes son llamados “mono excitados,” relativos al de referencia, el cual usualmente (pero no siempre) es el estado basal Hartree-Fock. El problema con este método  es la perdida de la extensividad del mismo.
\\


La energía total, como algunas otras propiedades de sistemas moleculares, es extensiva. Este es un término que proviene de la termodinámica y significa que si hay varios subsistemas son aislados unos de otros, debería ser posible obtener la energía total a través de calcular la energía de cada uno por separado y después sumarlas. Dentro de la química cuántica, un método es llamado extensivo de tamaño si cumplen esta propiedad. Sin embargo, tan pronto como la expansión CI es truncada a un nivel de excitación particular, se pierde la extensividad.

\subsection{MRCI}
Ocasionalmente, CI se trunca a niveles grandes de excitaciones, pero esto es muy caro y rara vez vale la pena. Se puede pensar que los niveles más altos de excitación son solo necesarios para algún subconjunto seleccionado de orbitales, este lleva a un CI multi referenciado, MRCI. El cual consiste en hacer excitaciones simples y dobles sobre todos los determinantes, pero tomando como referencia una función de onda multiconfiguracional. Comparado con el método SRCI, el número de configuraciones se incrementa por un factor aproximadamente igual al número de configuraciones incluidas en el MCSCF. Las funciones de onda MRCI que contienen un gran número de configuraciones pueden generar funciones de onda muy precisas, sin embargo, estas son computacionalmente caras de calcular.


\subsection{CASPT2}
Para estados de capa cerrada o capa abierta con alto spin, si pueden ser aproximados razonablemente con un solo determinante de Slater, hay métodos eficientes para incluir efectos de correlación dinámica, como teoría de perturbaciones (PT) o cumulos acoplados (CC). Hay variantes multiconfiguracionales de clúster acoplado, pero estos solo permiten utilizar muy pocas configuraciones.
\\


Un método popular es MP2 (teoría de perturbaciones de M\o ller–Plesset a segundo orden). Tratar de agregar efectos de correlación dinámica de una manera similar para sistemas multi configuracionales no es fácil. Sin embargo, en el caso de funciones de onda CASSCF, existe una aproximación similar a Moller–Plesset. CASPT2, el cual es un método para calcular la contribución de la correlación dinámica a segundo orden sobre CASSCF. Esto es con la teoría de perturbaciones de Rayleigh–Schrödinger, para una sola función de estado electrónico. Sin embargo, el estado no perturbado no es un estado de un solo determinante, pero si un estado CASSCF, típicamente incluyendo unas pocas cientos de miles de funciones de configuraciones de estado. Por lo tanto, cualquier tipo de estado (radical, cargado, excitado, etc.) puede ser usado, siempre que sea adecuadamente escrito por una función de onda CASSCF. El CASPT2 de multi estado puede trabajar simultáneamente con diferentes estados. La combinación CASSCF/CASPT2 ha resultado ser un gran excito para tratar una gran variedad de problemas. 
\newpage

\section{Efecto de Jahn-Teller de primer y segundo orden}

El efecto de Jahn-Teller se estudia realizando la respuesta de una molécula a una perturbación en la geometría, esto se realiza mediando una expansión en series de Taylor sobre la energía de la siguiente manera:


\begin{ceqn}
\begin{align}
\begin{split}
H=&H^{(0)}+\lambda H^{(1)} + \lambda^2 H^{(2)}+\dots \\
E(\lambda)=&E^{(0)}+\dfrac{\partial E}{\partial \lambda}\lambda+\dfrac{1}{2}\dfrac{\partial^2E}{\partial \lambda^2}\lambda^2+
\dfrac{1}{6}\dfrac{\partial^3E}{\partial \lambda^3}\lambda^3+\dots
\end{split}
\end{align}
\end{ceqn}



Para nuestro sistema, el hamiltoniano de orden cero que es el utilizado para calcular la energía de orden cero no es más que el hamiltoniano electrónico.

\begin{align}
H^{(0)}=h_{el}
\end{align}

La perturbación a primer orden consta de la primera derivada del sistema con respecto a la fuerza de la perturbación, que en este caso es el desplazamiento $\Delta q$ realizado sobre la geometría.

\begin{align}
H^{(1)}=\dfrac{\partial h_{el}}{\partial q} \cdot \Delta q
\end{align}

La perturbación a segundo orden es similar a la primera solo que en este caso se trata de la segunda derivada.

\begin{align}
H^{(2)}=\dfrac{\partial^2 h_{el}}{\partial q^2} \cdot \Delta q^2
\end{align}

\newpage
De esta manera la ecuación de las derivadas queda de la siguiente forma:
\\


\begin{ceqn}
\begin{align}
\begin{split}
E_i(\vec{q})= & E_i^0+\bra{\Psi_i}\dfrac{\partial h_{el}}{\partial q}\ket{\Psi_i}\cdot \Delta q +\bra{\Psi_i}\dfrac{\partial^2 h_{el}}{\partial q^2}\ket{\Psi_i}\cdot \Delta q^2\\
& + \sum	_{j(\neq i)} \dfrac{\abs{\bra{\Psi_i}\dfrac{\partial h_{el}}{\partial q}\ket{\Psi_j}}^2}{E_i-E_j}\cdot\Delta q^2
\end{split}
\end{align}
\end{ceqn}

El efecto de Jahn-Teller de primer orden puede ser explicado con el segundo término, ya que, sin entrar en muchos detalles, Arthur Jahn y Edward Teller\cite{JT} demostraron que para un estado degenerado no igualmente ocupado, la integral $\bra{\Psi_i}\dfrac{\partial h_{el}}{\partial q}\ket{\Psi_i}$ no puede ser cero por razones de simetría, esto puede ser visto como que el gradiente no sea cero en ese punto de alta simetría y se genere así un rompimiento espontaneo de la simetría de la molécula.
\\


El efecto de Jahn-Teller de segundo orden se entiende con el tercer y cuarto termino, ya que el cálculo de segundas derivadas se vuelve una competencia entre estos términos, el primer término $\bra{\Psi_i}\dfrac{\partial^2 h_{el}}{\partial q^2}\ket{\Psi_i}$   es un valor promedio al que siempre será positivo. 

\begin{center}
$\sum_{j(\neq i)} \dfrac{\abs{\bra{\Psi_i}\dfrac{\partial h_{el}}{\partial q}\ket{\Psi_j}}^2}{E_i-E_j}\cdot\Delta q^2$
\end{center}

El cuarto termino es el que incluye la interacción del estado estudiado con los demás estados excitados, el cual, de ser el estado basal provoca que este término sea negativo debido al denominador de la fracción. Si los estados electrónicos son cercanos en energía o si la interacción entre dichos estados es muy grande, este término crece, pudiendo provocar así que las segundas derivadas sean negativas, lo cual se ve reflejado en frecuencias imaginarias. Para el caso de estudiar un estado que no es el basal, la diferencia de energía se vuelve positiva con respecto al estado basal, esto provoca que las segundas derivadas sean positivas. Para nuestro caso el cual será desarrollado con mayor detalle más adelante, debido a la simetría del problema el efecto de Jahn-Teller de segundo orden se refleja como tres frecuencias imaginarias de simetría T$_{2g}$.




\subsection{Método de diagnóstico BBR}

El método propuesto por Bearpark, Blancafort y Robb \cite{Bearpark2002} se basa en un cálculo de frecuencias a nivel CASSCF con simetría restringida, y la evaluación del efecto de Jahn Teller de segundo orden para el caso de interés.
Esto mediante un cálculo CASSCF en el cual al calcular las constantes de fuerza incluyen un término importantes, que es la  contribucion de los vectores de rotacion $\dfrac{\partial E}{\partial\textbf{C}     \partial \mu} \textbf{C}_\lambda $ que contienen los elementos de acoplamiento derivativos $ \dfrac{\partial E}{\partial \textbf{C} \partial \mu} =\bra{\Psi_0}\dfrac{\partial H}{\partial Q_u}\ket{\Psi_k}$ los cuales son importantes ya que no solo incluye las constantes vibronicas, sino que también las contribuciones vibronicas de estados excitados que corresponden, estrictamente hablando, al efecto “real” de JT de segundo orden aparece debido a la mezcla de estados excitados. La contribución de este término al efecto de JT de segundo orden en general, puede ser determinado calculando las constantes de fuerza con y sin restricciones de simetría en el cálculo de derivadas de acoplamiento. 
\\



\subsection{Intersecciones cónicas y cruces evitados}

Las reglas de no cruzamiento garantizan que para una molécula diatónica los estados electrónicos de la misma simetría no puedan cruzarse. Esta regla falla para moléculas de más átomos, para las cuales los estados electrónicos de la misma simetría tienen permitido cruzarse, debido a la adición de más grados de libertad. Cuando los estados se cruzan, estos pueden formar una intersección cónica (IC), la cual es estrictamente de $(3n-6-2)$ dimensiones \cite{Applegate2003}. Una de las propiedades especiales de las IC, es que, la molécula puede pasar de una superficie electrónica a otra con gran facilidad. 

\newpage

\subsubsection{Una descripción matemática de las ICs}

Es instructivo construir sobre algunos conceptos para describir las superficies de energía potencial cercanas a intersecciones cónicas. Considerando un par de estados electrónicos (pueden tenerse más de dos estados, pero los principios son bien ilustrados solo con dos estados). La dependencia de las energías de los estados electrónicos sobre las $3N-6=M$ coordenadas vibracionales  $Q=(q_1,q_2,…,q_m )$ , puede ser representada por una matriz de 2 X 2 que representa el hamiltoniano electrónico $\mathcal{H}(Q)$, donde 

\begin{align}
\mathcal{H}(Q)= \bar{H}(Q)
\left(
\begin{matrix}
1 & 0  \\
0 & 1  \\
\end{matrix}
\right)
+R(Q)
\left(
\begin{matrix}
\cos \alpha & \sin \alpha  \\
\sin \alpha & -\cos \alpha \\
\end{matrix}
\right)
\label{ic7}
\end{align}

Donde $\bar{H}$ , $R(\Delta H y H_{12} )$, y $\alpha$ son funciones de $Q$ y están definidas como

\begin{align}
\bar{H}(Q)=\dfrac{1}{2}(H_{11}+H_{22}); \Delta H (Q)=\dfrac{1}{2(H_{11}-H_{22})}
\label{ic8}
\end{align}

\begin{align}
R(Q)=(\Delta H^2+H_{12}^2)^{\frac{1}{2}}
\label{ic9}
\end{align}

\begin{align}
\alpha(Q)= \cos^{-11}
\left(
\begin{matrix}
\dfrac{\Delta H }{R}  \\
\end{matrix}
\right)
=\sin^{-1}
\left(
\begin{matrix}
\dfrac{H_{12}}{R}  \\
\end{matrix}
\right)
\end{align}

Si bien estas ecuaciones pueden aparentar ser innecesariamente complicadas, los parámetros fueron elegidos debido a su significado físico. Los elementos diagonales del Hamiltoniano $H_{11}$ y $H_{22}$ tienen como eigenfunciones las dos funciones de base adiabáticas, $\phi_1$  y $\phi_1$. La fuerza de la interacción entre  $\phi_1$  y $\phi_1$ es determinada por la separación energética, $\Delta H$, y por la magnitud del acoplamiento vibrónico, $H_{12}$.


\newpage
Los eigenvalores generales, $E_\pm$ y los eigenfunciones, $\psi_\pm$, de $H(Q)$, son

\begin{align}
E_{\pm}(Q)=\bar{H}\pm R
\end{align}

\begin{align}
\psi_+=\cos \left( \dfrac{\alpha}{2} \right) \phi_1 
+ \sin \left( \dfrac{\alpha}{2} \right) \phi_2
\end{align}

\begin{align}
\pi_-=-\sin \left( \dfrac{\alpha}{2} \right) \phi_1 
+ \cos \left( \dfrac{\alpha}{2} \right) \phi_2
\end{align}

Donde $\psi_\pm$ son comúnmente llamados estados diabáticos, ya que la transformación a ellos causa que los elementos no adiabáticos se desvanezcan. Los estados diabáticos de la molécula se cruzarán cuando las energías de los estados se vuelven degeneradas, para esas geometrías se satisface $E_+ (Q)=E_- (Q)$, que requiere

\begin{align}
\bar{H}+R=\bar{H}-R
\label{ic14}
\end{align}

Utilizando las ecuación \ref{ic14} con las ecuaciones \ref{ic8}-\ref{ic9}, encontramos el par de condiciones sobre las cuales dos estados se intersecan en la geometría $Q^0$,

\begin{align}
\Delta H (Q^0)=0
\label{ic15}
\end{align}

\begin{align}
\Delta H_{12}(Q^0)=0
\label{ic16}
\end{align}

Donde si la intersección es o no cónica no depende de otros factores, que explicaremos adelanto.

\newpage

\subsubsection{Caso 1: dos estados que no se cruzan dentro de la aproximación de Born-Oppenheimer.}

Si los estados electrónicos están bien separados en energía ($\Delta H$ es grande), entonces una buena aproximación de los eigenvalores de $\mathcal{H}(Q)$, la ecuación \ref{ic7}, se simplifican a 

\begin{align}
E_+=\bar{H}+\Delta H = E_1  ; \ \psi_+=\phi_1
\label{ic17}
\end{align}

\begin{align}
E_-=\bar{H}-\Delta H = E_2 ;\ \psi_-=\phi_2
\label{ic18}
\end{align}

Esto es, que los estados diabáticos son los mismos que los estados adiabáticos.  Esta situación ocurre cuando la aproximación del Born-Oppenheimer es válida. De hecho, la mayoría de las moléculas de capa cerrada con estados electrónicos bien separados se describen más que adecuadamente dentro de la aproximación de Born-Oppenheimer, al menos cerca de las geometrías de equilibrio.

\subsubsection{Caso 2: una intersección de estados no cónica.}

Es posible que diferentes estados electrónicos de una molécula puedan cruzarse sin formar una IC, como es el cruce entre dos estados de diferente simetría. En este caso, el elemento de interacción $H_{12}$ se desvanece por simetría para todas las opciones de $Q$.  La intersección ocurre en geometrías  $Q^0$ que solo necesitan satisfacer la ecuación \ref{ic15} y es por lo tanto de dimensión $M-1$.

\subsubsection{Caso 3: una intersección cónica general}
Si las ecuaciones \ref{ic15} y \ref{ic16} se cumplen simultáneamente por razones diferentes a las mencionadas antes. Entonces una intersección cónica de dimensión $M-2$ ocurre. Estas condiciones pueden ser satisfechas gracias a la simetría o debido a un accidente o casualidad. (Por casualidad significa que existe una $Q^0$ para la cual las ecuaciones \ref{ic15} y \ref{ic16} se cumplen, pero no hay condiciones de simetría que determinen esos valores.) Las intersecciones cónicas que se encuentran en coordenadas de reacciones comúnmente se deben a la casualidad. Debido a que las ecuaciones \ref{ic15} y \ref{ic16} son independientes puede darse el caso en que una se cumpla por simetría y otra se cumpla por casualidad. Un ejemplo de esto\cite{Applegate2003}, es el caso de los dos estados de más baja energía $^1A"$ y $^2A"$ del H$_2$S, el cual tiene una intersección cónica donde la ecuación \ref{ic15} se cumple debido a la simetría de la geometría, $C_{2v}$. Sin embargo, la ecuación \ref{ic16} se cumple por casualidad.


\subsubsection{Caso 4: intersecciones cónicas por Jahn-Teller
} 

El caso en el que las ecuaciones \ref{ic15} y \ref{ic16}. son satisfechas por simetría es conocida como el efecto de Jahn-Teller. Una condición del teorema de Jahn-Teller es que es necesario tener una simetría molecular con un eje de rotación  $C_3$ o de orden mayor para generar representaciones irreducibles degeneradas del grupo puntual. Además, la simetría particular de la molécula dicta la simetría de las coordenadas ($\tau$,$\tau_2$) que componen la intersección cónica.  Jahn y Teller demostraron que la simetría de ($\tau$,$\tau_2$) además corresponde a una vibración de la molécula, la cual es doblemente degenerada. (Para los grupos de simetría cúbicos, existen degeneraciones mayores que 2, pero aquí hablamos de los casos más simples.) Para una molécula con Jahn-Teller, existirán p vibraciones que tienen la misma simetría que ($\tau$,$\tau_2$ ). Al ser doblemente degeneradas estas vibraciones deben de tener en cuenta $2p$ grados de libertad, Sin embargo, la intersección cónica se genera de solo dos grados de libertad.

\begin{figure}[h]
\centering
\includegraphics[trim={0cm 4cm 1cm 2cm},clip,scale=0.4]{figuras/JT.pdf} 
\caption{Representación de una intersección cónica en el caso del efecto de Jahn-Teller }
\end{figure}

\newpage
\subsubsection{Caso 5: el efecto de Jahn-Teller de segundo orden}

La naturaleza es raramente discontinua, por lo que algunas veces se debe esperar una transición entre la situación donde la aproximación de Born-Oppenheimer es válida (caso 1) y cuando es totalmente invalida, la intersección cónica de Jahn-Teller (caso 4.) Matemáticamente esto ocurre cuando $\Delta H$ en las ecuaciones \ref{ic17} y \ref{ic18} se acerca, pero no llega a cero. Un valor finito para $H_12$ en la ecuación \ref{ic9} puede afectar significativamente a la superficie de energía potencial, $E_\pm$. Dicha situación es conocida cono el efecto de pseudo Jahn-Teller o efecto de Jahn-Teller de segundo orden. Si $E_\pm$ se aproxima de cerca a una IC pero en realidad diverge antes de alcanzar el cruce, corta la superficie mostrando un cruce evitado que puede tener efectos profundos en la dinámica molecular. El efecto sobre las superficie de energía potencial de cualquier acoplamiento vibrónico no adiabático entre diferentes estados electrónicos, pueden ser referidos como efecto de pseudo de Jahn-Teller, incluso si $\Delta H$ no es particularmente pequeño.

\begin{figure}[h]
\centering
\includegraphics[trim={0.2cm 2cm 1cm 2cm},clip,scale=0.4]{figuras/JT2.pdf} 
\caption{Representación de una cruce evitado generado por el efecto Jahn-Teller de segundo orden }
\end{figure}

\newpage


\chapter{Metodología}

Los cálculos realizados para este estudio fueron principalmente CASSCF, este es un cálculo que realiza una interacción de configuraciones completa en un espacio activo definido, este espacio activo esta dado a través de un número de electrones y orbitales. Los orbitales que son de nuestro interés son los orbitales de valencia los cuales son 8. El número de electrones activos es 8,9 o 10 para las especies H$_4$X$^{2+}$, H$_4$X$^+$ y H$_4$X respectivamente. Al estudiar una geometría tetraédrica $T_d$ podemos separar los orbitales en dos conjuntos a1 y dos conjuntos $t_2$. Los cálculos actuales en los que trabajamos utilizan el subgrupo abeliano $D_2$ en el cual los estaos $T2$ quedan separados en $B_1+B_2+B_3$. Para un cálculo de frecuencias sin simetría el número total de configuraciones generadas en los cálculos CASSCF para las especies H$_4$X$^{2+}$, H$_4$X$^+$, H$_4$X son 1764, 2352 y 1176, respectivamente. 
\\


Para el caso del H$_4$S$^+$, también probamos incluir un mayor número de orbitales en el espacio activo. A través de una inspección sobre las energías de los orbitales a nivel ROHF, podemos saber que después de los orbitales moleculares que se correlacionan con los orbitales atómicos 3s, 3p del azufre vienen los 4s,4p, y con una energía mayor el conjunto de los orbitales 3d. Las propiedades químicas del azufre son comúnmente asociadas a la presencia de los orbitales 3d, por esta razón los espacios activos más grandes que estudiamos incluían los orbitales 4s,4p y 3d en adición del espacio de valencia completo de 8 orbitales. En este caso el número de configuraciones obtenidas para el cálculo de frecuencias es de 6311760 sin simetría.
\\


La interpretación usual del efecto de Jahn-Teller de segundo orden para un estado basal de alta simetría, es que existen términos de acoplamientos vibrónicos entre el estado de interés y los estados excitados de la misma simetría que la frecuencia vibracional inestable (imaginaria). Por ejemplo, para las especies tetraédricas H$_4$X (radical o capa cerrada) el desarrollo a primer orden de la función de onda del estado basal tiene la forma: 

\begin{align*}
\Psi(Q_{t2})= \ket{A_1}+Q_{t_2} \cfrac{\bra{A_1}\partial H/ \partial Q_{t_2} \ket{T_2} }{E_{A_1}-E_{T_2}} \ket{T_2}+ \dots
\end{align*}

\newpage	

Y los términos de la energía cuadrática en $Q_{t_2}$ que son sospechosos de originar la inestabilidad con respecto a la distorsión t2 tienen la forma:

\begin{align}
\cfrac{\abs{\bra{A_1} \partial H / \partial Q_{t_2} \ket{T_2} }^2}{E_{A_1}-E_{T_2}}
\label{55}
\end{align}

Donde $\partial H / \partial Q_{t_2} $representa una derivada del Hamiltoniano con respecto a una coordenada geométrica $t_2$ y las E son las energías de los estados electrónicos, el término \ref{55} es negativo debido a que el estado A$_1$ es el estado basal.
Para llevar a cabo un análisis sobre la posible presencia del efecto de Jahn-Teller de segundo orden aplicamos el procedimiento desarrollado por Bearpark, Blancafort, y Robb \cite{Bearpark2002}, el cual está implementando en Gaussian09\cite{Gaussian}. Este es un método comúnmente utilizado cuando un cálculo de frecuencias estándar revela una frecuencia imaginaría en un punto de alta simetría. El cálculo se repite, pero las configuraciones que pueden originar el efecto de Jahn-Teller se remueven del espacio activo. La energía electrónica se mantiene intacta, pero las curvaturas de los modos de interés se vuelven más positivas debido a que los términos a segundo orden que son negativos han sido removidos de la expansión a la energía. La desaparición de las curvaturas negativas es considerada una manifestación del efecto de Jahn Teller en el cálculo original de frecuencias. En nuestro caso concreto este método consiste en excluir las configuraciones $T_2 (B_1+B_2+B_3)$ del cálculo de segundas derivadas y verificar que el valor de la frecuencia de los modos imaginarios $T_2$ se vuelve un valor real.
\\


Debido a que el efecto de Jahn-Teller esta comúnmente asociado a la presencia de estados excitados de baja energía, es necesario tratar con la misma importancia tanto el estado estudiado, como los estados con los que se puede acoplar. Esto no se puede lograr con métodos comunes, donde las optimizaciones de parámetros variacionales de la función de onda se concentran en el estado de interés. Incluso en los cálculos CASSCF donde se optimizan los orbitales, estos se concentran en un solo estado y no se hace un balance con todos los estados electrónicos relevantes. Esta es la razón por la que realizamos cálculos con un promedio de estados (SA). Por ejemplo, para los radicales H$_4$X$^+$ y NH$_4$ en el estado $^2A_1$, realizamos cálculos estándar y con promedio de estados con el mismo espacio activo, pero promediando con iguales pesos el estado basal $^2A_1$ y los estados $^2T_2$ más bajos. Como se mostrará en la sección de resultados, el promedio de estados proporciona un mejor tratamiento para el acoplamiento vibrónico de estados que el cálculo de CASSCF estándar.


Cuando analizamos las curvaturas de los modos $T_2$ de las estructuras NH$_4$ y H$_4$O$^+$, utilizamos la siguiente descomposición de la energía electrónica total con respecto a los desplazamientos geométricos $t_2$.

\begin{align}
E(Q_{t_2})=E_{closed-shell}^{core}(Q_{t_2})+VAE(Q_{t_2})
\label{koopmans57}
\end{align}

Donde $E_{closed-shell}^{core}$ es la energía del core de capa cerrada y $VAE$ es la energía electrónica vertical de captura. La separación de $E_{closed-shell}^{core}$ es ventajosa porque existe evidencia para esperar que este término tenga una curvatura positiva para NH$_4^+$ y H$_4$O$^{2+}$. Por lo que cualquier diferencia cualitativa entre estos radicales debe depender de la curvatura del término VAE. Los términos $E_{closed-shell}^{core}$  y $VAE$ pueden ser calculados con cualquier modelo de estructura electrónica y las curvaturas totales para los casos difíciles, en particular para el H$4$S$^+$, están siendo estudiadas a niveles más avanzados de teoría. Aquí, sin embargo, adaptaremos la ecuación \ref{koopmans57} al modelo más simple de estructura electrónica, el modelo del teorema de Koopmans’\cite{Koopmans}

\begin{align}
(Q_{t_2})=E_{closed-shell}^{RHF,core}(Q_{t_2})+\epsilon_{3a_1}(Q_{t_2})
\label{koopmans58}
\end{align}

Donde la energía del core de capa cerrada a nivel RHF se complementa con el energía del orbital LUMO. El LUMO es $3a_1$ en la simetría $T_d$ y se vuelve $4a_1$ una vez echa la distorsión en $t_2$ para volverse $C_{3v}$, y nosotros monitorizamos la energía del orbital a lo largo de la distorsión $t_2$ en la forma de un diagrama de Walsh.\cite{Walsh}

\newpage

\section{Herramientas computacionales}

\subsection{Hardware}

Este trabajo fue realizado con recursos computacionales del laboratorio de supercómputo del Centro de Investigaciones Químicas, estos recursos incluyen varios clústeres compartidos, siendo estos Jumil con 32 procesadores AMD y 64 Gb de RAM, Moyote con 64 procesadores AMD y 128 Gb de RAM, y Mantis de 64 procesadores AMD y 256 Gb de RAM, a su vez se utilizó una computadora de escritorio de 4 procesadores AMD y 8 Gb de RAM.

\subsection{Software}
Para este trabajo todos los cálculos fueron  realizados mediante los softwares Gaussian 09 \cite{Gaussian} y Molpro\cite{Molpro} . Las gráficas y algunas de las figuras se obtuvieron con el graficador Gnuplot\cite{Gnuplot}, los diagramas de contorno y dibujos de orbitales moleculares fueron obtenidas utilizando Molden\cite{Molden} y multiwfn\cite{Multiwfn}, las fracciones de electrones y valores de contorno correspondientes fueron cálculos con OpenCubeMan\cite{opencubeman}. Chemcraft\cite{Chemcraft} y Gauss View\cite{Gaussian} se utilizaron para visualizar las geometrías, orbitales, modos de vibración, pasos de optimizaciones y pasos en las coordenadas de reacción.
\newpage

\chapter{Resultados}
Para comenzar, era necesario validar la metodología midiendo la precisión de nuestros cálculos del tipo CASSCF con espacio activo completo, en las tablas \ref{tr1}-\ref{tr4} reportamos las geometrías y frecuencias para las especies H$_2$O, H$_3$O$^+$, H$_2$S, H$_3$S$^+$, H$_2$Se las cuales a su vez se compararan con los valores experimentales. La concordancia resulto ser bastante buena, con un porcentaje de error menor que un 4\% para las geometrías y menor que un 7\% para las frecuencias. También es importante notar que existe una pequeña dependencia de las propiedades con el conjunto de base utilizado.


\begin{table}[ht!]
\centering
\resizebox{\textwidth}{!}{
\begin{tabular}{l|lll|lll|lll}
\hline
 Molécula &  & H$_2$O &  &  & H$_2$S &  &  & H$_2$Se  &  \\ \hline
\begin{tabular}[c]{@{}l@{}} Conjunto \\ de base\end{tabular}& Exp\cite{Cook1974} & AWCVDZ & AWCVTZ & Exp\cite{Edwards2004} & AWCVDZ & AWCVTZ  &  Exp\cite{Oka1962} & AWCVTZ & AWCVDZ \\
 R(\AA)& 0.96 & 0.97 & 0.96 & 1.34 & 1.36 & 1.35 & 1.46 & 1.48 & 1.48 \\
 Ángulo& 103.9 & 102.6 &103.0 & 92.1 &93.0  & 93.1 & 90.9 &92.3  &  92.2 \\ \hline
\end{tabular}
}
\caption{Geometrías de equilibrio nivel CASSCF utilizando 8 electrones y 8 orbitales activos, de las especies H$_2$X(X=O,S,Se) y la comparación con el experimento. 
}
\label{tr1}
\end{table}

\begin{table}[h]
\centering
\resizebox{\textwidth}{!}{
\begin{tabular}{l|lll|lll|lll}
\hline
 Molécula &  & H$_2$O &  &  & H$_2$S &  &  & H$_2$Se  &  \\ \hline
\begin{tabular}[c]{@{}l@{}} Conjunto \\ de base\end{tabular}& Exp\cite{Shimanouchi2009} & AWCVDZ & AWCVTZ & Exp\cite{Shimanouchi2009} & AWCVDZ & AWCVTZ  &  Exp\cite{Shimanouchi2009} & AWCVDZ & AWCVTZ \\
A$_1$ & 3657 & 3752 & 3749 &2615  & 2609 & 2611 & 2345 & 2353 & 2344 \\
A$_1$ & 1595 & 1687 & 1690 &1183  & 1175 &1181  & 1034 & 1042 & 1038 \\
B$_1$ & 3756 & 3870 & 3861 &2626  & 2634 &2634  & 2358 & 2370 & 2360 \\ \hline
\end{tabular}
}
\caption{Frecuencias vibracionales armónicas en cm$^{-1}$ calculadas a nivel CASSCF utilizando 8 electrones y 8 orbitales activos, para las especies H$_2$X(X=O,S,Se) y la comparación con el experimento.
}
\end{table}


\newpage


\begin{table}[H]
\centering
\begin{tabular}{l|lll|lll}
\hline
Molécula &  & H$_2$O &  &  & H$_2$S &  \\ \hline
\begin{tabular}[c]{@{}l@{}} Conjunto \\ de base\end{tabular}&
Exp\cite{Tang1999} & AWCVDZ & AWCVTZ & Exp\cite{Nakanga1989} & AWCVDZ & AWCVTZ  \\
R(\AA) & 0.97  & 0.99 & 0.98 & 1.35 & 1.37 & 1.36 \\
Angulo & 113.6  & 108.8 & 109.4 &94.2  & 94.5 &94.6  \\ \hline
\end{tabular}
\caption{Geometrías de equilibrio nivel CASSCF utilizando 8 electrones y 8 orbitales activos, de las especies H$_3$X$^+$ (X=O,S) y la comparación con el experimento}
\end{table}

\begin{table}[H]
\centering
\begin{tabular}{l|lll|lll}
\hline
Molécula &  & H$_2$O &  &  & H$_2$S &  \\ \hline
\begin{tabular}[c]{@{}l@{}} Conjunto \\ de base\end{tabular}&
Exp \cite{Tang1999} & AWCVDZ & AWCVTZ & Exp\cite{Nakanga1989} & AWCVDZ & AWCVTZ  \\
E & 3519 & 3621 & 3618 & 2526 & 2557 & 2563 \\
E & 3519 & 3621 & 3618 & 2526 & 2557 & 2563 \\
A1& 954  & 1086 & 1076 & 1033 & 1060 &  1060\\
E & 1639 & 1709 & 1734 & -    & 1171 &1174  \\
E & 1639 & 1709 & 1734 & -    & 1171 & 1174 \\
A1& 3491 & 3517 & 3531 & 2521 & 2539 & 2550 \\ \hline
\end{tabular}
\caption{Frecuencias vibracionales armónicas en cm$^{-1}$ calculadas a nivel CASSCF utilizando 8 electrones y 8 orbitales activos, para las especies H$_3$X$^+$ (X=O, S) y la comparación con el experimento.}
\label{tr4}
\end{table}

El efecto de aumentar el espacio activo incluyendo los orbitales atómicos 3d, 4s, 4p del azufre tiene únicamente un pequeño efecto sobre las propiedades generales del tetraedro H$_4$S$^+$. las distancias SH decrecen por menos de un 1\%. Solamente una frecuencia T2 se mantienen imaginaria, y esta muestra una fuerte dependencia con el conjunto de base utilizado, dicha dependencia se discutirá a en este trabajo. De ahora en adelante, todos los cálculos CASSCF (SS o SA) para todas las especies químicas se realizará con 8 orbitales y entre 8-10 electrones en el espacio activo. 

\newpage

\section{Estabilidad local de la especie tetraedrcia H$_4$X$^{2+}$}
Empezaremos a discutir el caso del H$_4$X$^{2+}$ en el estado basal 1A1. Para todos los calcógenos encontramos un mínimo local, consistente con los estudios previos para el oxígeno y azufre. En la figura \ref{rdicationes} y tabla \ref{rdicationest} mostramos las frecuencias armónicas calculas a nivel CASSCF/aug-cc-pwCVTZ para los calcógenos. Estas pueden dividirse en frecuencias pequeñas de doblamiento E y T$_2$, y frecuencias más altas de modos de estiramiento A$_1$ y T$_2$. Se puede observar un patrón monótonamente decreciente para todos los valores de las frecuencias según se baja en la tabla periódica. También es importante notar la cuasi degeneración entre las frecuencias de estiramiento A$_1$ y T$_2$ exceptuando los casos del oxígeno y el polonio que parecen apartarse ligeramente de este patrón.
\\


\begin{figure}[h]
\centering
\includegraphics[trim={1cm 1cm 1cm 1cm},clip,scale=0.7]{figuras/dicationes.png} 
\caption{	Frecuencia vibracional para todas las especies dicationicas H$_4$X$^{2+}$ (X= O, S, Se, Te, Po) a nivel CASSCF/aug-cc-pwCVTZ.}
\label{rdicationes}
\end{figure}


\begin{table}[h]
\centering	
\begin{tabular}{l|l|l|l|l|l}
\hline
 Molécula & H$_4$O$^{2+}$ & H$_4$S$^{2+}$ & H$_4$Se$^{2+}$ &  H$_4$Te$^{2+}$ & H$_4$Po$^{2+}$ \\ \hline
R(\AA) & 1.045 & 1.390 & 1.505 & 1.678 & 1.760 \\ 
 &  &  &  &  &  \\ 
 t2 & 1469 & 978 & 854 & 721 & 643 \\ 
e &   1649 & 1163&1021 & 838 & 758 \\ 
t2 & 2847 & 2382 & 2215 & 2036 & 1891 \\
a1 & 2977 & 2367 & 2198 & 2009 & 1837 \\ \hline
\end{tabular}
\caption {Frecuencia vibracional para todas las especies dicationicas H4X2+ (X= O, S, Se, Te, Po) a nivel CASSCF/aug-cc-pwCVTZ.}
\label{rdicationest}
\end{table}



\section{Efecto de Jahn-Teller de segundo orden para la especie tetraédrica H$_4$X$^+$ y el caso particular del H$_4$S$^+$}

La primera parte del estudio consistió en analizar el tetraedro H$_4$X$^+$ en el estado basal $^2$A$_1$. Para este caso hay un incremento en la distancia de enlace XH en comparación con el H$_4$X$^{2+}$ desde 0.05\AA $ $ para él O hasta 0.18\AA $ $ para él Te. Además, para todos los calcógenos, se predijo una inestabilidad en la estructura $T_d$ con una frecuencia vibracional imaginaría para los modos de estiramiento T$_2$. En la figura \ref{rmonocationes} y tabla \ref{rmonocationest} se muestran las frecuencias vibracionales para todos los monocationes calculados a nivel CASSCF/aug-cc-pwCVTZ de un solo estado (SS). Lo más sorprendente es la fuerte variación en la frecuencia T$_2$ imaginaria, en particular es importante notar que el valor de la frecuencia imaginaria para el caso del azufre es de 6558 i cm$^{-1}$ el cual no parece tener sentido físico, cosa que será mejor desarrollada más adelante en esta tesis.  Para los modos de estiramiento A$_1$ y de doblamiento E, las frecuencias se reducen entre 400-500 cm$^{-1}$ en comparación con el caso del dication de capa cerrada, y la frecuencia de doblamiento T$_2$ es la menos afectada.



\begin{figure}[h]
\centering
\includegraphics[trim={1cm 1cm 1cm 1cm},clip,scale=0.5]{figuras/figura2.png} 
\caption{Frecuencias harmónicas vibracionales en cm$^{-1}$ para los monocationes H$_4$X$^+$ (X= O, S, Se, Te, Po) en el estado 2A1 calculadas con la base aug-cc-pwCVTZ. a)  SS CASSCF. b) SA CASSCF: donde se promediaron con los mismos pesos el estado basal $^2$A$_1$ y los primeros estados excitados $^2$T$_2$.}
\label{rmonocationes}
\end{figure}


\begin{table}[H]
\centering	
\begin{tabular}{l|l|l|l|l|l}
\hline
 Molécula & H$_4$O$^{2+}$ & H$_4$S$^{2+}$ & H$_4$Se$^{2+}$ &  H$_4$Te$^{2+}$ & H$_4$Po$^{2+}$ \\ \hline
R(\AA) & 1.093 & 1.535 & 1.671 & 1.856 & 1.927 \\ 
 &  &  &  &  &  \\ 
 t2 & 1924$i$ & 6885$i$ & 3275$i$& 2417$i$ & 1257$i$ \\ 
e &   1063 & 382 & 460 & 441 & 492 \\ 
t2 & 1191 & 964 & 931 & 900 & 980 \\
a1 & 2441 & 1725 & 1630 & 1518 & 1465 \\ \hline
\end{tabular}
\caption {Frecuencias harmónicas vibracionales en cm$^{-1}$ para los monocationes H4X+ (X= O, S, Se, Te, Po) en el estado 2A1 a nivel SS CASSCF/aug-cc-pwCVTZ.}
\label{rmonocationest}
\end{table}



\begin{table}[H]
\centering	
\begin{tabular}{l|l|l|l|l|l}
\hline
 Molécula & H$_4$O$^{2+}$ & H$_4$S$^{2+}$ & H$_4$Se$^{2+}$ &  H$_4$Te$^{2+}$ & H$_4$Po$^{2+}$ \\ \hline
R(\AA) & 1.120 & 1.560 & 1.699 & 1.889 & 1.966 \\ 
 &  &  &  &  &  \\ 
 t2 & 1304$i$ & 3679$i$ & 2442$i$& 1855$i$ & 1113$i$ \\ 
e &   1109 & 397 & 436 & 415 & 428 \\ 
t2 & 1162 & 928 & 885 & 847 & 906 \\
a1 & 2440 & 1599 & 1535 & 1415 & 1392 \\ \hline
\end{tabular}
\caption {Frecuencias harmónicas vibracionales en cm$^{-1}$ para los monocationes H$_4$X$^+$ (X= O, S, Se, Te, Po) a nivel SA CASSCF/aug-cc-pwCVDZ promediando los estado $^2$A$_1$ y $^2$T$_2$}
\label{rmonocationest2}
\end{table}


La predicción de una frecuencia imaginaría para los monocationes H$_4$X$^+$ se mantiene para los cálculos SA CASSCF, donde el estado basal $^2$A$_1$ y el excitado $^2$T$_2$ son promediados con los mismos pesos, como se puede ver en la figura \ref{rmonocationest2}. Un conjunto de modos T$_2$ siguen teniendo unas frecuencias imaginarías altas, sin embargo, estas son considerablemente menores que las de los cálculos estándar. Por ejemplo, la frecuencia T$_2$ para el azufre se redujo por 3229 $i$ cm$^{-1}$ a un valor más sensato de 3679 $i$ cm$^{-1}$.  La reducción análoga para el oxígeno y selenio es de 620 y 832 i cm$^{-1}$ respectivamente. Las frecuencias para el segundo conjunto T2 no mostraron una fuerte dependencia entre el cálculo estándar y el de promedio de estados, la diferencia no excedió los 100 cm$^{-1}$. Las frecuencias de los modos E y A$_1$ son siempre más grandes para el cálculo estándar que para el cálculo SA, con discrepancias de entre 100-200 cm$^{-1}$ para el modo A$_1$ y de menos de 100 cm$^{-1}$ para los modos E.
\\


Para caracterizar aún mejor el origen de las frecuencias imaginarias aplicamos el diagnóstico de Bearpark, Blancafort, y Robb (BBR). Basándonos en un cálculo CASSCF para X=O, S, y Se. En la figura \ref{rmonocationes2} y tabla \ref{rmonocationest3} reportamos las frecuencias T$_2$ del cálculo estándar y las mismas después de realizar el procedimiento diagnóstico. La presencia del efecto de Jahn-Teller de segundo orden es evidenciado debido al gran cambio en el conjunto de frecuencias de estiramiento T$_2$, de imaginarias a reales, una vez que los estados $^2$T$_2$ los cuales se acoplan vibronicamente con el estado basal $^2$A$_1$ se remueven del espacio activo. El segundo conjunto de frecuencias T$_2$, las cuales son reales, apenas se ven modificadas por el procedimiento BBR. Para apoyar esta interpretación, visualizamos los modos virbacionales antes y después de aplicar el procedimiento BBR concluyendo que las frecuencias imaginarias están asociadas a modos de estiramiento, cuyas curvaturas cambian de negativo a positivo, mientras que el segundo conjunto T$_2$ están asociados a modos de doblamiento cuyas frecuencias apenas se ven afectadas por el procedimiento. Por supuesto, las frecuencias A$_1$ y E se mantienen sin cambios tras la eliminación de los estados $^2$T$_2$. Gracias a estas pruebas, se pudo concluir que las frecuencias imaginarias que aparecen en el monocatión H$_4$X$^+$ son debido al efecto de Jahn-Teller de segundo orden.

\begin{figure}[h]
\centering
\includegraphics[trim={1cm 1cm 1cm 0cm},clip,scale=0.6]{figuras/figura3.png} 
\caption{Frecuencias harmónicas vibracionales en cm$^{-1}$ para el monocatión H4X+ (X= O, S, Se) a dos niveles:  a) SS CASSCF estándar (S) b) CASSCF estándar con el procedimiento BBR (BBR). Utilizando la base aug-cc-pwCVTZ.}
\label{rmonocationes2}
\end{figure}

\newpage

\begin{table}[]
\centering
\begin{tabular}{l|l|l|l}
\hline
 Molécula &H$_4$O$^+$  & H$_4$S$^+$ & H$_4$Se$^+$ \\ \hline
R(\AA) & 1.093 & 1.535 & 1.671 \\ \hline
 &  &  &  \\
t2 &1501  & 922 & 829 \\ 
e & 1192 & 423 & 461 \\ 
t2 & 4114 & 2417 & 2194 \\ 
a1 & 2441 & 1707 &  1628 \\ \hline
\end{tabular}
\caption{Frecuencias harmónicas vibracionales en cm$^{-1}$ para los monocationes H$_4$X$^+$ (X= O, S, Se) a nivel CASSCF estándar con el procedimiento BBR (BBR). Utilizando la base aug-cc-pwCVTZ.}
\label{rmonocationest3}
\end{table}




La dependencia de los valores de las frecuencias imaginarias con respecto al conjunto de base es importante en la aproximación SS, de nuevo, los efectos más grandes están presentes para el azufre con un incremento de 1212 i cm$^{-1}$ ante el cambio de base de aug-cc-pwCVDZ a aug-cc-pwCVTZ. El cambio análogo para SA CASSCF es de 221 $i$ cm$^{-1}$, el cual todavía es significativo. La frecuencia del segundo conjunto T2 no muestra una dependencia grande con los métodos SS vs SA o con el conjunto de base, las discrepancias no exceden los 100 cm$^{-1}$. A su vez, las frecuencias de los modos E y A$_1$ dependen débilmente del métodos o del conjunto de base utilizada. Las mayores diferencias son para el oxígeno, las cuales son sobre 200 cm$^{-1}$, pero sin excederse a los 100 cm$^{-1}$ para los otros calcógenos. Es importante notar, que en el caso del oxígeno las frecuencias E y A$_1$ son mucho más grandes que para el resto de los calcógenos, así que la diferencia porcentual es pequeña para toda la serie de calcógenos.


La inusual sensibilidad para las frecuencias de estiramiento T$_2$ en el H$_4$S$^+$ con respecto al método y al conjunto de base, al igual que el intervalo inusual de las frecuencias imaginarias (3500-6900 i cm$^{-1}$) nos obligó a examinar más a detalle este sistema. Los métodos convencionales de estructura electrónica de bajo nivel, con un solo determinante (UHF, UMP2) pueden predecir una curvatura incorrecta para este modo y los valores de las frecuencias resultan muy dispersos. Es necesario un tratamiento extensivo para la correlación electrónica para enmendar los resultados incorrectos al evaluar las frecuencias harmónicas vibracionales. El tratamiento SA es superior al tratamiento SS, no solo para CASSCF, sino que también cuando utilizamos métodos multireferenciales altamente correlacionados. Algunos funcionales bien establecido de intercambio y correlación han sido probados y estos se desenvuelven mucho mejor que los métodos UHF o UMP2, ya que estos siempre predicen frecuencias imaginarias de magnitudes razonables. 


\begin{table}[h]
\centering
\begin{tabular}{l|l|l|l|l}
\hline
Método & HF & HF-SA & CCSD & CCSD-SA \\ \hline
t2 & 5980.50 & 2688.48 $i$ & 1857.5 & 4148.46 $i$ \\
e & 649.87 &  503.38	& 538.3	 & 488.57 \\
t2 & 1240.72 &1224.93  & 276.1& 1138.08 \\
a1 & 2031.54 & 1917.65 & 1878.2 & 1897.72 \\ \hline
\end{tabular}
\caption{Frecuencias harmónicas vibracionales en cm$^{-1}$ para el monocatión H4S+ con diferentes métodos SS y SA y la base. aug-cc-pwCVDZ}
\label{monocationessa}
\end{table}


\begin{table}[H]
\centering
\begin{tabular}{l|l|l|l|l|l}
\hline
Método & RS2 & RS2-SA & RS2-SA-LS & MRCI & MRCI-SA \\ \hline
t2 &9659.37  &3326.13  &1856.37$i$  & 833.41$i$ & 3795.83$i$ \\
t2 &9659.37  &3326.13  & 1856.37$i$  &185.83  & 3795.83$i$  \\
t2 &9659.37  &3326.13  &1856.37$i$   & 1022.69 & 3795.83$i$  \\
E  &722.02  & 644.92 & 545.62 &1087.06  & 447.85 \\
E  &722.02  & 644.92 & 545.62 & 1580.76 & 447.85 \\
t2 &1152.68  &984.07 & 1136.67 & 2021.36 & 1076.92 \\
t2 &1152.68  &984.07  & 1136.67 & 7859.39 &1076.92  \\
t2 &1152.68  & 984.07 &1136.67  & 7930.96 & 1076.92 \\
a1 &1698.71  &1902.67 & 1821.93 & 7931.93 &1798.04  \\ \hline
\end{tabular}
\caption{Frecuencias harmónicas vibracionales en cm$^{-1}$ para el monocatión H$_4$S$^+$ en el estado $^2$A$_1$ con diferentes métodos multireferenciales altamente correlacionados y la base. aug-cc-pwCVDZ}
\label{monocationessa2}
\end{table}

\newpage
Como se puede ver en las tablas \ref{monocationessa} y \ref{monocationessa2}, el tratamiento SA mejora cualquier tipo de cálculo, desde los monoreferenciados hasta los multireferenciados altamente correlacionados, el único caso anómalo fue el de RS2. Hacer SA sobre este cálculo arregla el problema de obtener valores anormalmente grandes, pero sigue teniendo un error al momento de identificar las frecuencias imaginarias. Para arreglar dicho problema se realizó un cálculo de desplazamiento de estados (LS), en este cálculo arbitrariamente se desplazan las energías de los llamados estados intrusos, los cuales son estados cercanos en energía que pueden causar ruido al momento de realizar algunos cálculos, esto arreglo el problema. 

\section{Estabilidad de la especie tetraedrcia NH$_4$ vs inestabilidad del H$_4$O$^+$}

Comenzaremos con los cores de capa cerrada isoelectricos, NH$_4^+$ y H$_4$O$^{2+}$. Típicamente los modos de estiramiento OH tienen frecuencias más altas que los modos de estiramiento NH. Por ejemplo, las frecuencias de los modos de estiramiento en el agua son de entre 200-300 cm$^{-1}$ más grandes que en el amoniaco. Los modos de estiramiento T$_2$ y A$1$ en el NH$_4^+$ están, sin embargo, entre 300-600 cm$^{-1}$ corridos hacia el azul en comparación con los modos de estiramiento en el H$_4$O$^{2+}$. Nosotros interpretamos esta diferencia como una manifestación de la estabilidad local del H$_4$O$^{2+}$ y como consecuencia de “sobrecargarlo” con protones. El H$_4$O$^{2+}$ es instable con respecto a H$_3$O$^{2+}$ + H$^+$ por 61.3 kcal/mol, a través de un estado de transición $C_{3v}$ con un enlace OH elongado a 2.06 \AA $ $ y con una barrera considerable de 38.1 kcal/mol. El NH$_4^+$, por otro lado, es globalmente estable, lo cual se manifiesta con una afinidad electrónica significativa del NH$_3$ de 204.0 kcal/mol. En cuanto a los modos de doblamiento, las frecuencias son de nuevo menores en el H$_4$O$^{2+}$ que en el NH$_4^+$, pero las diferencias no exceden los 120 cm$^{-1}$.

\begin{figure}[h!]
\centering
\includegraphics[trim={0cm 1cm 1cm 0cm},clip,scale=0.4]{figuras/4a.pdf} 
\caption{Frecuencias harmónicas vibracionales en cm$^{-1}$ para las especias NH$_4^+$ y H$_4$O$^{2+}$ a dos niveles:  a) SS CASSCF estándar (S) b) CASSCF estándar con el procedimiento BBR. Utilizando la base aug-cc-pwCVTZ.}
\label{rmonocationes2}
\end{figure}

\begin{table}[h!]
\centering
\begin{tabular}{l|ll|ll}
\hline
Molécula &    H$_4$N$^+$ &  &    H$_4$O$^{2+}$ &  \\ \hline
Conjunto de base &  \multicolumn{1}{l|}{AWCVTZ} & AWCVTZ-BBR  & \multicolumn{1}{l|}{AWCVTZ} & AWCVTZ-BBR \\ \hline
R(\AA) & \multicolumn{1}{l|}{1.030} & 1.030 & \multicolumn{1}{l|}{1.045} & 1.045 \\
 & \multicolumn{1}{l|}{} &  & \multicolumn{1}{l|}{} &  \\
t2 & \multicolumn{1}{l|}{1508} & 2438 & \multicolumn{1}{l|}{1469} & 2519 \\
e & \multicolumn{1}{l|}{1768} & 1768 & \multicolumn{1}{l|}{1649} & 1650 \\
t2 & \multicolumn{1}{l|}{3438} & 4740 & \multicolumn{1}{l|}{2847} & 4771 \\
a1 & \multicolumn{1}{l|}{3314} & 3314 & \multicolumn{1}{l|}{2977} & 2977 \\ \hline
\end{tabular}
\caption{Frecuencias harmónicas vibracionales en cm$^{-1}$ para las especias NH$_4^+$ y H$_4$O$^{2+}$ a dos niveles:  a) SS CASSCF estándar (S) b) CASSCF estándar con el procedimiento BBR (BBR). Utilizando la base aug-cc-pwCVTZ.}
\end{table}


\newpage
\textcolor{white}{Hola}
\newpage
Al unir un electrón al core de capa cerrada, el NH$_4$ en el estado basal $^2$A$_1$ experimenta una elongación pequeña del enlace NH por 0.005 \AA, la estructura tetraédrica se mantiene como mínimo y las frecuencias de estiramiento NH se reducen por 200 cm$^{-1}$. En comparación con los del NH$_4^+$. El tetraedro H4O+ en el mismo estado 2A1 experimenta una elongación más significativa de los enlaces OH por 0.05 \AA, y desarrolla una curvatura negativa para los modos de estiramiento T$_2$. Como se discutió en la sección 1.5.3, esta curvatura negativa se desarrolla como consecuencia del acoplamiento vibrónico en el estado electrónico $^2$T$_2$, el cual está separado del estado 2A1 por 24063 cm$^{-1}$ en nuestros cálculos SA CASSCF. Puede ser sorprendente que la separación entre los estados $^2$A$_1$ – $^2$T$_2$ para el NH$_4$ es incluso más pequeña, 12409 cm$^{-1}$, pero el estado 2A1 se mantiene como un mínimo y el tratamiento BBR incrementa los modos de estiramiento T$_2$ por tan solo 1168 cm$^{-1}$. El efecto es mucho más pequeño que en el H4O+, en el cual la eliminación de las CSF’s  $^2$T$_2$ cambian la frecuencia de 1924 i cm$^{-1}$ a 4114 cm$^{-1}$. La importancia del acoplamiento vibrónico, es definitivamente más pequeña en el NH$_4$ que en el H$_4$O$^+$.

\begin{table}[h]
\centering
\begin{tabular}{l|l|l}
\hline
CASSCF (9,8) & H4N  & H$_4$O$^+$  \\ \hline
$^2$A$_1$ & -56.78122 & -76.80499 \\
$^2$T$_2$ & -56.71895 & -76.69405 \\
$\Delta$ E – (UA)  & 0.06227225 & 0.11094 \\
$\Delta$ E – (cm$^{-1}$) &13667.2  & 35032.3\\ \hline
\end{tabular}
\caption{Diferencia de energía entre los estrados $^2$A$_1$ y $^2$T$_2$ para el H$_4$N y H$_4$O$^+$.}
\end{table}

\newpage

Las curvaturas de estiramiento T$_2$ en el NH$_4$ y H$_4$O$^+$ mantienen signos contrarios a través de varios niveles de teoría, incluyendo cálculos no restringidos mono referenciados, desde UHF hasta CCSD. Para interpretar las diferencias cualitativas en las curvaturas T$_2$, expresaremos la energía total del radical, como la suma de las energías del core de capa cerrada y la energía de enlace, Ecuación 58, y discutiremos el resultado numérico utilizando el modelo del teorema de Koopmans’, ya que este se acopla de la misma manera que la energía de core de capa cerrada y la del enlace del electrón. Monitorizando la energía LUMO del core de capa cerrada a lo largo de una distorsión t2.
\\


La distancia XH (X=N, O) se incrementó gradualmente por 0.3 \AA$ $ desde el mínimo de $T_d$ de capa cerrada y los demás grados de libertad se minimizaron hasta resultar en una secuencia de estructuras $C_{3v}$. Un scan relajado sorbe la superficie de energía potencial para el core de capa cerrada se realizó a nivel RHF/aug-cc-pwCVTZ. Estas geometrías se utilizaron para monitorear la dependencia de $E_{(closed-shell)}^{(RHF, core)}$ y $\epsilon_{(3a_1)}$ sobre la distorsión t2, y el resulto se ilustra en la figura \ref{walsh}, donde a y b se utilizan para NH$_4$ y H$_4$O$^+$, respectivamente.

\begin{figure}[h!]
\centering
\includegraphics[trim={0cm 0cm 0cm 0cm},clip,scale=0.4]{figuras/figura5.png} 
\caption{Diagrama de Walsh para el a) NH$_4$ y b) H$_4$O$^+$, calculado a nivel RHF/aug-cc-pwCVTZ.}
\label{walsh}
\end{figure}


\newpage
Las diferencias cualitativas sobre las curvaturas de los estiramientos T$_2$ resultan de la combinación de dos efectos: (i) La energía del orbital LUMO decrece más rápido en el H$_4$O$^{2+}$ que en el NH$_4^+$ a lo largo de la distorsión t2, y (ii) el H$_4$O$^{2+}$ tiene una curvatura más pequeña que el NH4+. El decrecimiento precipitado de $\epsilon_{3a_1}$ en el H$_4$O$^{2+}$ es responsable de la curvatura negativa T$_2$ en el radical H$_4$O$^+$. Un decrecimiento gradual de $\epsilon_{3a_1}$ en el NH$_4^+$ deja al radical $T_d$ geométricamente estable localmente, aunque la frecuencia T$_2$ es aproximadamente 200 cm$^{-1}$ más pequeña en el NH$_4$ que en el NH$_4^+$.
\\


Las energías de orbitales moleculares de alta ocupación son comúnmente graficados como función de desplazamientos geométricos, para justificar cual es la estructura molecular predilecta. En algunos casos está claro desde la simetría del orbital de interés y el resultado de las interacciones de enlace/antienlace entre los orbitales atómicos dominantes involucrados, y en si una energía de un orbital molecular particular incrementa o decrece como función del desplazamiento geométrico. En nuestro caso, sin embargo, $\epsilon_{3a_1}$ decrese a lo largo de la distorsión t2 para ambos radicales. Es este ratio de decrecimiento y la curvatura del core de capa cerrada, los que deciden acerca de la estabilidad de la estructura del radical tetraédrico.
\\


Uno debe tener en mente que las energías de los orbitales involucradas en el diagrama de Walsh envuelve tres componentes: la componente Coulombica atractiva con los núcleos $V_{ne}$, la aproximación del campo promedio de la repulsión electrónica $V_{ee}$, y el termino cinético t, dos de esos términos, $V_{ne}$ y Vee pueden ser un orden de
 magnitud más grandes en valor absoluto que las energías de los orbitales, Para el H$_4$O$^{2+}$, $\epsilon_{3a_1}$ es de -12.1 eV pero los términos $V_{ne}$, $V_{ee}$, y $t$ son       -132.2, +96.1, y +23.9 eV, respectivamente. De manera similar, para el NH$_4^+$, $\epsilon_{3a_1}$ es de -4.0 eV, pero los componentes correspondientes son -75.7, +63.8, y 8.0 eV,
 respectivamente. Las gráficas de estos términos individuales como función de las distorsión t2 se muestran en la figura \ref{walsh2}.

\newpage 
 
El patrón de cambios es el mismo para el caso del N y del O: Vne se vuelve más atractiva, Vee se vuelve más repulsiva, y el termino $t$ incrementa. El efecto general es dictado por el termino Vne. Se encuentra un decrecimiento en el valor de $\epsilon_{3a_1}$ para ambos NH$_4^+$ y H$_4$O$^{2+}$. Este patrón general es el mismo para el N y para el O donde los valores absolutos de los cambios individuales se vuelven relevantes. Estos son sistemáticamente mayores para este último y el decrecimiento de $\epsilon_{3a_1}$ en el H$_4$O$^{2+}$ es pronunciado hasta el punto donde la curvatura T$_2$ se vuelve negativa en el H$_4$O$^+$.

\begin{figure}[h!]
\centering
\includegraphics[trim={0cm 1.5cm 0cm 0cm},clip,scale=0.5]{figuras/figura6.pdf} 
\caption{Comparación entre los términos involucrados en el diagrama de Walsh, Vne, Vee, $\epsilon_{3a_1}$ y t, entre NH$_4$ y H$_4$O$^+$}
\label{walsh2}
\end{figure}

Volviendo a analizar las diferencias en los orbitales moleculares del NH$_4$ y H$_4$O$^+$ que se producen debido a la diferencia entre la carga nuclear ente el N y el O, y que se ven reflejadas sobre todo en la carga +e y +2e del core de capa cerrada al que está unido el electrón en el radical. Por supuesto, se espera que los orbitales de este último sean más compactos.  De hecho, los orbitales 3a1 en el NH$_4^+$ y en el H$_4$O$^{2+}$ difieren significativamente en la extensión radial. El nodo más externo se interseca con los enlaces OH, y en el caso del N se encuentra fuera del core molecular. Esta característica es consistente con el hecho de que la distancia OH aumenta considerablemente cuando un electrón se enlaza al core de capa cerrada, pero no la distancia NH. 
La diferencia en la extensión espacial se ve reflejada en el dominio de las contribuciones de los orbitales atómicos. El orbital $3_{a1}$ en el NH$_4^+$ tiene una contribución significativa de las funciones de base s más difusas sobre el N y el H, mientras que la contribución de los orbitales 1s de los H es muy pequeña. En el orbital $3_{a1}$ del H$_4$O$^{2+}$, por otro lado, la última contribución es muy importante y comparable con la contribución del 2s del O. 	También es necesario considerar el orbital LUMO+1, 2$t_2$, debido a que a lo largo de la distorsión la t2 su componente paralelo al eje $C_3$ puede mezclarse con el LUMO debido a que ambos de vuelven a1 en la simetría $C_{3v}$. 
\\


El orbital $2t_2$ tiene contribuciones importantes de los 1s de los hidrógenos en el caso del O, pero no en el del N. Se puede justificar la diferencia en las composiciones LCAO del $3_{a1}$ y del $2_{t2}$ en el NH$_4^+$ y en el H$_4$O$^{2+}$ en términos de las diferencias en electronegatividades del N y el O, a través de estos sistemas pude ser más justificable comparar N y O$^+$, y las diferencias son incluso más profundas. La mezcla de estos dos orbitales está permitida por simetría (a1 en $C_{3v}$), lo cual ayuda a la localización del LUMO en el átomo de hidrogeno que sale cuando se combina con las modificaciones del potencial de campo promedio, y debido a esto es el responsable de la disminución de la energía del orbital, como se observa en la figura \ref{orbitales}. Según aumenta la distorsión $T_d \rightarrow C_{3v}$, El SOMO del radical se vuelve el orbital 1s del átomo de hidrogeno saliente con una energía orbital de -13.6 eV. 
\\


La transformación del LUMO del core de capa cerrada al orbital 1s del hidrogeno es mucho más rápido en el H$_4$O$^{2+}$ que en el NH$_4^+$ debido a que la deslocalización del LUMO y LUMO+1 en el primero tienen contribuciones importantes del 1s. En adición, el electrón desapareado en el hidrogeno saliente es estabilizado por la interacción coulombica con el H$_3$O$^+$ que se queda detrás, pero para el caso del nitrógeno la interaccione es repulsiva con el NH$_3$. Creemos que estos factores contribuyen al decrecimiento más pronunciado de $\epsilon_{3a_1}$ en el H$_4$O$^{2+}$ sobre la distorsión t2.


\begin{figure}[h!]
\centering
\includegraphics[trim={0cm 1.5cm 0cm 0cm},clip,scale=0.5]{figuras/figura7.pdf} 
\caption{Diagrama de contorno del H$_4$N y H$_4$O$^+$ incluyendo los valores correspondientes de fracciones de electrón desde 90\% a 10\%. }
\label{orbitales}
\end{figure}

\newpage

A este punto tenemos dos interpretaciones sobre la diferencia cualitativa de las frecuencias T$_2$ entre H$_4$O$^+$ y NH$_4$. La primera es el acoplamiento vibrónico entre los estados electrónicos $^2A_1$ y $^2T_2$, confirmado por el diagnostico BBR en la sección 6.2 y la segunda basada en las ecuaciones \ref{koopmans57}-\ref{koopmans58}, y la VAE aproximada mediante $\epsilon_{3a_1}$. La sencillez de estos sistemas nos permite identificar la relación entre ambas interpretaciones. La interpretación de los orbitales ya fue discutida anteriormente por lo que nos centraremos en el acoplamiento vibrónico basándonos en la expansión de la función de onda del radical sobre una la distorsión t2 desde $T_d$. Las funciones
$\ket{^2A_1}$ y$\ket{^2T_1}$ usadas en la ecuación \ref{55} puede ser aproximada por los determinantes principales (o configuraciones de funciones de estado):

\begin{ceqn}
\begin{align}
\begin{split}
\ket{^2A_1}=\text{det}|core \quad 3a_1|+\cdots\\
\ket{^2T_2}=\text{det}|core  \quad 2_{t_2}|+\cdots
\end{split}
\end{align}
\end{ceqn}

\newpage
Donde se puede suponer que los orbitales de core y los orbitales de core doblemente ocupados de capa cerrada son los mismos en ambos determinantes (modelo de Koopmans’).

\begin{align*}
\Psi	(Q_{t2})=\text{det}|core \quad 3a_1|+\text{det}|core  \quad 2_{t_2}|+\cdots=\text{det}|core (\quad 3a_1 + c 2_{t_2} )| + \cdots
\end{align*}


Donde el coeficiente c depende de la magnitud de la distorsión t2, la fuerza del acoplamiento vibrónico entre $^2A_1$ y $^2T_2$, y la diferencia de energía entre estos estados electrónicos, ecuación \ref{55}. La parte derecha de la ecuación \ref{koopmans58} ilustra que la función de onda del radical está dominada por el determinante único con el orbital SOMO como resultado de la interacción de los estados $3a_1$ y $2_{t_2}$.


\section{Estabilidad de la especie tetraedrcia NH$_4^-$ vs inestabilidad del H$_4$X (X=calcogeno)}


El siguiente paso, es comparar los tetraedros NH$_4^-$ y H$4$X en el estado fundamental $^1$A$_1$. Solo el NH4- ha sido observado experimentalmente\cite{Beyer1977}. De hecho, es un mínimo en la superficie de energía potencial, como se puede ver en la figura \ref{neutras} y tabla \ref{neutrast}. La presencia del exceso de electrones no “debilita” el marco molecular. Primero, no hay elongación de los enlaces NH en comparación NH$_4$. Segundo, todas las frecuencias de vibración son más grandes en el NH$_4^-$ que, en el NH$_4$, y además mas pequeñas que en el NH$_4^+$. La sorprendente estabilidad del NH$_4^-$ está relacionada con un carácter muy difuso del orbital $3a_1$. De hecho, el tetraedro NH$_4^-$ fue apodado como un anión doble de Rydberg\cite{Beyer1977} para reflejar las características únicas de su estructura electrónica. 
\\


La estabilidad prometedora del tetraedro NH$_4^-$ podría sugerir que al intercambiar el N con el O neutralizaría el sistema y por lo tanto aliviaría la inestabilidad ingrediente de Coulomb que cualquier sistema cargado posee. Sin embargo, el H$_4$O desarrollo frecuencias imaginaries T$_2$ que exceden 1300 i cm$^{-1}$, y el diagnostico BBR revelo la importancia del acoplamiento vibrónico con los estados electrónicos $^1$B$_2$. El enlace OH se elongan por 0.08 \AA y todos los modos vibracionales son más suaves en comparación con los del H$_4$O$^+$.
\\

\newpage

La curvatura negativa de los modos T$_2$ persiste en los calcógenos más pesados H$_4$X. En adición, los modos de doblamiento E desarrollan una curvatura negativa en el caso del S, Se y Te. Para el H$_4$Po, la curvatura E es positiva pero la frecuencia es menor que 200 cm$^{-1}$. El conjunto de frecuencias A$_1$ y el segundo conjunto T$_2$ son reales, pero siempre más pequeñas que en el caso H$_4$X$^+$ por 100-600 cm$^{-1}$. La distancia XH se incrementa por 0.12 \AA (Po) y 0.15 \AA (S, Se, Te) en comparación con H$_4$X$^+$. Todos estos son indicios de que la estabilidad local del H$_4$X$^{2+}$ se rompe con la reducción de carga positiva con uno y dos electrones. Mientras que para el caso del NH$_4^+$ no hay perdida de estabilidad geométrica con uno o dos electrones, de hecho la estructura del NH4 es más rígida que la del NH$_4$.



\begin{figure}[h]
\centering
\includegraphics[trim={0cm 2cm 0cm 2cm},clip,scale=0.4]{figuras/4c.pdf} 
\caption{Frecuencias harmónicas vibracionales en cm$^{-1}$ para las especias NH$_4^-$ y H$_4$O a dos niveles:  a) SS CASSCF estándar (S) b) CASSCF estándar con el procedimiento BBR (BBR). Utilizando la base aug-cc-pwCVTZ.}
\label{neutras}
\end{figure}

\begin{table}[h]
\centering
\begin{tabular}{l|ll|ll}
\hline
Molécula &    H$_4$N$^-$ &  &    H$_4$O &  \\ \hline
Conjunto de base &  \multicolumn{1}{l|}{AWCVTZ} & AWCVTZ-BBR  & \multicolumn{1}{l|}{AWCVTZ} & AWCVTZ-BBR \\ \hline
R(\AA) & \multicolumn{1}{l|}{1.034} & 1.034 & \multicolumn{1}{l|}{1.159} & 1.159 \\
 & \multicolumn{1}{l|}{} &  & \multicolumn{1}{l|}{} &  \\
t2 & \multicolumn{1}{l|}{1404} & 4311 & \multicolumn{1}{l|}{1384 $i$} &  3405 \\
e & \multicolumn{1}{l|}{1733} & 1733 & \multicolumn{1}{l|}{863} & 863 \\
t2 & \multicolumn{1}{l|}{3304} & 1788 & \multicolumn{1}{l|}{999} & 909 \\
a1 & \multicolumn{1}{l|}{3122} & 3122 & \multicolumn{1}{l|}{1804} & 1804 \\ \hline
\end{tabular}
\caption{recuencias harmónicas vibracionales en cm$^{-1}$ para las especias NH$_4^-$ y H$_4$O a dos niveles:  a) SS CASSCF estándar (S) b) CASSCF estándar con el procedimiento BBR (BBR). Utilizando la base aug-cc-pwCVTZ.}
\label{neutrast}
\end{table}

\newpage
\chapter{Conclusiones y Perspectivas}

Mediante La validación del método se mostró que los cálculos CASSCF con base aug-cc-pwCVTZ, eran capaces de reproducir de buena manera los resultados experimentales, por lo que pudimos concluir que el método era lo suficientemente preciso para reproducir las propiedades fisicoquímicas de estas especies. 
\\


La estabilidad de los dicationes H$_4$X$^{2+}$ geométrica fue comprobada para todos los calcógenos, confirmando los resultados obtenidos en trabajos previos para el H$_4$O$^{2+}$ y H$_4$S$^{2+}$. Todos los dicationes resultaron ser inestables termodinámicamente, pero estables cinéticamente ante la reacción de pérdida de un protón, dichos resultados serán reportados en un trabajo a futuro.
\\



Se logro desarrollar una metodología para determinar la inestabilidad de los monocationes H$_4$X$^+$, ya que encontramos que los problemas que tenían los diferentes tipos de cálculos para obtener las derivadas era debido a que los estados $^2$A$_1$ y $^2$T$_2$ son muy cercanos en energía, y por lo tanto no pueden ser tratados por separado correctamente, si no que tienen que ser tomados en cuenta ambos en un cálculo de promedio de estados para obtener los resultados correctos, posterior a esto se rastreó la inestabilidad de los monocationes al efecto de Jahn-Teller de segundo orden, esto siendo confirmado con el método de diagnóstico BBR.. A su vez se comparó el H$_4$O$^+$ con el NH$_4$ el cual es una especie bioelectrónica a los monocationes, pero estable, las diferencias que rastreamos para que esta especie sea estable son, que el acoplamiento vibrónico entre los estados $^2$A$_1$ y $^2$T$_2$ es mucho más débil en el caso del nitrógeno y que además existen diferencias clave en los orbitales SOMO, ya que el del nitrógeno es mucho mas difuso y presenta un mayor carácter de antienlace y de Rydberg, lo que ayuda a su estabilización, cosa que en el oxígeno no ocurre.
\\


Las especies H$_4$X también fueron estudiadas, aunque estas, similar al caso del H$_4$X$^+$, son inestables debido al efecto de Jahn-Teller de segundo orden. También se comparó el H$_4$O con el NH4- obteniendo resultados equivalente al caso del H$_4$O$^+$ con el NH$_4$.
\\


En este trabajo se presentó poco sobre los cálculos de promedio de estados altamente correlacionado, esto debido a que es un trabajo en proceso y no ha sido terminado, además se estudió la estabilidad cinética y termodinámica de les especies H$_4$X$^{2+}$ con respecto a diferentes procesos disociativos, estos dos trabajos al ser terminados se esperan que en un futuro sean discutidos y presentados en un artículo científico.

\newpage
\addcontentsline{toc}{chapter}{Bibliografía}
\begin{thebibliography}{13}
\bibitem {Karkamar2009}	Karkamkar A, Kathmann SM, Schenter GK, et al. Thermodynamic and structural investigations of ammonium borohydride, a solid with a highest content of thermodynamically and kinetically accessible hydrogen. Chem Mater. 2009;21(19):4356-4358. doi:10.1021/cm902385c
 
\bibitem {Olah1986}  Olah GA, Surya Prakash GK, Barzaghi M, Lammertsma K, Schleyer P von R, Pople JA. Protonated Hydronium Dication, H402+. Hydrogen-Deuterium Exchange of D2H17O+in HF:SbF5 and DH2170+ in DF:SbF5 and Theoretical Calculationsla. J Am Chem Soc. 1986;108(5):1032-1035. doi:10.1021/ja00265a031 

 
\bibitem {Olah1988} 	Olah GA, Prakash GKS, Marcelli M, Lammertsma K. The tetrahydridosulfonium dication, H4S2+. Hydrogen-deuterium exchange of DH2S+ in FSO3D:SbF5 and D2HS+ in FSO3H:SbF5 and theoretical calculations. J Phys Chem. 1988;92(4):878-880. doi:10.1021/j100315a005


\bibitem {Kozmuta1982}  Kozmutza C, Kapuy E, Robb MA, Daudel R, Csizmadia IG. Theory of lone pairs. IV. Molecular ion hole states of ten-electron hydrides. Molecular ionization potentials and proton affinities by directSCF calculations. J Comput Chem. 1982;3(1):14-22. doi:10.1002/jcc.540030104

\bibitem {Choi1988}  Choi SC, Boyd RJ, Knop O.  A 6-31G* chemistry of isoelectronic tetrahedral XL 4 $\epsilon$ and YL 4 $\epsilon$ (X = Li to F; Y = Na to Cl; L = H, F, Cl) species. Part 2. Energies, bond lengths, and critical radii . Can J Chem. 1988;66(9):2465-2475. doi:10.1139/v88-388

\bibitem {Boldyrev1992}  Boldyrev AI, Simons J. Ab initio study of geometrically metastable multiprotonated species: MH  k +   n . J Chem Phys. 1992;97(6):4272-4281. doi:10.1063/1.463929


\bibitem {Bearpark2002}  Bearpark MJ, Blancafort L, Robb MA. The pseudo-Jahn-Teller effect: A CASSCF diagnostic. Mol Phys. 2002;100(11):1735-1739. doi:10.1080/00268970110105442

\bibitem {Herzberg1984}  	Herzberg G. Spectra of the ammonium radical: the schüler bands. J Astrophys Astron 1984 52. 1984;5(2):131-138. doi:10.1007/BF02714985

\bibitem{Boldyrev2002} 	Boldyrev AI, Simons J. Rydberg bonding in ammonium dimer ((NH4)2). J Phys Chem. 2002;96(22):8840-8843. doi:10.1021/J100201A029

\bibitem{Nilles2002} Xu, S. J. Nilles, J. M. Hendricks, J. H. Lyapustina, S. A. Bowen, K. H. Double Rydberg anions: Photoelectron spectroscopy of NH4-, N2H7-, N3H10-, N4H13-, and N5H16-. J Chem Phys. 2002;117(12):5742. doi:10.1063/1.1499491




\bibitem {Stoyanov2012} Stoyanov ES, Gunbas G, Hafezi N, et al. The R3O+ $\dotsi$ H+Hydrogen Bond: Toward a Tetracoordinate Oxadionium(2+) Ion. Published online 2012.


\bibitem {Schneider2010}  Schneider TF, Werz DB. Caged Chalcogens: Theoretical Studies on a Tetracoordinated Oxonium Dication and Its Higher Homologues. Org Lett. 2010;12(4):772-775. doi:10.1021/OL902904Z


\bibitem {Parry2002}  Parry RW, Schultz DR, Girardot PR. The Preparation and Properties of Hexamminecobalt(III) Borohydride, Hexamminechromium(III) Borohydride and Ammonium Borhydride1. J Am Chem Soc. 2002;80(1):1-3. doi:10.1021/JA01534A001

\bibitem {Alonso}  Alonso Jacobo-Hernández. Estudio teoríco de las especies doblemente protonadas:el caso del H4S2+. Published online 2019.

\bibitem{Ramon} A. 	Jacobo-Hernández  , B. Cox, S. Ling, R. Hernández-Lamoneda MG. Trabajo En Proceso.

\bibitem {Pauncz2018}  Pauncz R. The Symmetric Group in Quantum Chemistry. Symmetric Gr Quantum Chem. Published online May 4, 2018. doi:10.1201/9781351077224

\bibitem {Roos2016} Roos BO, Lindh R, Malmqvist P, Veryazov V, Widmark PO. Multiconfigurational Quantum Chemistry. Multiconfigurational Quantum Chem. Published online August 15, 2016:1-224. doi:10.1002/9781119126171


\bibitem {JT}  H. A. Jahn and E. Teller. Stability of polyatomic molecules in degenerate electronic states - I—Orbital degeneracy. Proc R Soc London Ser A - Math Phys Sci. 1937;161(905):220-235. doi:10.1098/RSPA.1937.0142


\bibitem {Applegate2003} Applegate BE, Barckholtz TA, Miller TA. Explorations of conical intersections and their ramifications for chemistry through the Jahn-Teller effect. Chem Soc Rev. 2003;32(1):38-49. doi:10.1039/a910269h

\bibitem{Jensen} Jensen F. Introduction to computational chemistry. Published online 2007:599. Accessed February 16, 2022. https://www.wiley.com/en-mx/Introduction+to+Computational+Chemistry\%2C+2nd+Edition-p-9780470058046


\bibitem {Gaussian} M. J. Frisch, G. W. Trucks, H. B. Schlegel, G. E. Scuseria, M. A. Robb, J. R. Cheeseman, G. Scalmani, V. Barone, G. A. Petersson, H. Nakatsuji, X. Li, M. Caricato, A. Marenich, J. Bloino, B. G. Janesko, R. Gomperts, B. Mennucci, H. P. Hratchian, J. V. Ort  and DJF. Gaussian 09. Published online 2016. http://gaussian.com/


\bibitem {Koopmans}  Koopmans T. Über die Zuordnung von Wellenfunktionen und Eigenwerten zu den Einzelnen Elektronen Eines Atoms. Physica. 1934;1(1-6):104-113. doi:10.1016/S0031-8914(34)90011-2

\bibitem{Walsh} Walsh AD. 466. The electronic orbitals, shapes, and spectra of polyatomic molecules. Part I. AH2 molecules. J Chem Soc. 1953;(0):2260-2266. doi:10.1039/JR9530002260

\bibitem{Molpro}H.-J. Werner, P. J. Knowles, G. Knizia, F. R. Manby, M. Schütz, P. Celani, W. Györffy, D. Kats, T. Korona, R. Lindh, A. Mitrushenkov, G. Rauhut, K. R. Shamasundar, T. B. Adler, R. D. Amos, S. J. Bennie, A. Bernhardsson, A. Berning, D. L. Cooper, M. J. O. MW. MOLPRO. Published online 2018. https://www.molpro.net/


\bibitem{Gnuplot} Janert PK.
Gnuplot in action :
understanding data with graphs.

\bibitem{Molden}MOLDEN a visualization program of molecular and electronic structure. Accessed August 12, 2021. https://www3.cmbi.umcn.nl/molden

\bibitem{Multiwfn}Lu T, Chen F. Multiwfn: A multifunctional wavefunction analyzer. J Comput Chem. 2012;33(5):580-592. doi:10.1002/JCC.22885

\bibitem{opencubeman}Haranczyk M, Gutowski M. Visualization of Molecular Orbitals and the Related Electron Densities. J Chem Theory Comput. 2008;4(5):689-693. doi:10.1021/CT800043A

\bibitem{Chemcraft}Andrienko. GA, And S thanks to INS for advice, Romanov  to A. ChemCraft. Published online 2018. https://www.chemcraftprog.com

\bibitem{Cook1974}Cook RL, De Lucia FC, Helminger P. Molecular force field and structure of water: Recent microwave results. J Mol Spectrosc. 1974;53(1):62-76. doi:10.1016/0022-2852(74)90261-6

\bibitem{Edwards2004} Edwards TH, Moncur NK, Snyder LE. Ground‐State Molecular Constants of Hydrogen Sulfide. J Chem Phys. 2004;46(6):2139. doi:10.1063/1.1841014

\bibitem{Oka1962}Oka T, Morino Y. Analysis of the microwave spectrum of hydrogen selenide. J Mol Spectrosc. 1962;8(1-6):300-314. doi:10.1016/0022-2852(62)90030-9

\bibitem{Shimanouchi2009}Shimanouchi T, Matsuura H, Ogawa Y, Harada I. Tables of molecular vibrational frequencies. J Phys Chem Ref Data. 2009;7(4):1323. doi:10.1063/1.555587

\bibitem{Tang1999}Tang J, Oka T. Infrared Spectroscopy of H3O+: The $\nu$ 1 Fundamental Band. J Mol Spectrosc. 1999;196(1):120-130. doi:10.1006/JMSP.1999.7844

\bibitem{Nakanga1989}Nakanaga T, Amano T. Difference frequency laser spectroscopy of SH3+: A simultaneous analysis of the $\nu$1 and $\nu$3 fundamental bands. J Mol Spectrosc. 1989;133(1):201-216. doi:10.1016/0022-2852(89)90254-3

\bibitem{Beyer1977} Beyer HK, Jacobs PA, Uytterhoeven JB, Till F. Thermal stability of NH4-chabazite. J Chem Soc Faraday Trans 1 Phys Chem Condens Phases. 1977;73(0):1111-1118. doi:10.1039/F19777301111



\end{thebibliography}
\end{document}