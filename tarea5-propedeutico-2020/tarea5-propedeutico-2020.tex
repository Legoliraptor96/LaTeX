\documentclass{article}
\usepackage[utf8]{inputenc}
\usepackage{amsmath}
\usepackage{amssymb}
\usepackage{graphicx}
\usepackage{epstopdf}
\usepackage{inputenc}
\usepackage{ textcomp }
\usepackage{geometry} 
\usepackage[document]{ragged2e}
\usepackage{fancyhdr}
\usepackage{enumerate}
\setlength{\baselineskip}{25pt}

\title{Curso Propedéutico DMNQ}
\author{Tarea 5}
\date{Septiembre 2020}


\begin{document}

\maketitle      

\section*{Resolver las siguientes ecuaciones:}

\begin{enumerate}
    \item $5y + 6y - 81 = 7y + 102 + 65y$
    \item $71 + [-5x + (-2x + 3)] = 25 - [-(3x + 4) - (4x - 3)]$
    \item $(x + 1)(2x + 5) = (2x + 3)(x - 4) + 5$
\end{enumerate}


\section*{Resuelva los siguientes problemas:}


\begin{enumerate}
    \item Repartir 133 pesos entre A, B y C de modo que la parte de A sea la mitad de la de B
y la de C sea el doble de la de B.
    \item La suma de tres números es 72. El segundo es $\frac{1}{5}$ del tercero y el primero excede al tercero en 6. Hallar los números
    \item En una clase hay 60 alumnos entre hombres y mujeres. El número de mujeres excede en 15 al doble de los hombres. Encuentra el número de mujeres y hombres en la clase
\end{enumerate}


\section*{Resolver las siguientes ecuaciones:}

\begin{enumerate}
    \item $10x - \frac{x-3}{4} = 2(x - 3)$
    \item $\dfrac{5x+8}{3x+4}=\dfrac{5x+2}{3x-4}$
    \item $\dfrac{5}{1+x} -\dfrac{3}{1-x} -\dfrac{6}{1-x^2} = 0$
    \item $a(x + b) + x(b - a)=2b(2a - x)$
\end{enumerate}

\section*{Resuelva los siguientes problemas:}

\begin{enumerate}
    \item  El largo de un barco, que mide 800 pies, excede en 744 pies a los $\frac{8}{9}$ del ancho. Encuentrael ancho del barco.
    \item  En 4 días un hombre recorrió 120 km. Si cada día recorrió $\frac{1}{3}$ de lo que recorrió el día anterior, ¿cuántos km recorrió en cada día?
    \item El denominador de una fracci´on excede al doble del numerador en 1. Si el numerador se aumenta en 15 y el denominador se disminuye en 1, el valor de la fracción es $\frac{4}{3}$. Hallar la fracción.
\end{enumerate}

\end{document}