\documentclass{article}
\usepackage[utf8]{inputenc}
\usepackage{amsmath}
\usepackage{amssymb}
\usepackage{graphicx}
\usepackage{epstopdf}
\usepackage{inputenc}
\usepackage{ textcomp }
\usepackage{geometry} 
\usepackage[document]{ragged2e}
\usepackage{fancyhdr}
\usepackage{enumerate}
\setlength{\baselineskip}{25pt}

\title{Curso Propedéutico DMNQ}
\author{Tarea 7}
\date{Septiembre 2020}


\begin{document}

\maketitle

\section*{Resolver los siguientes sistemas de ecuaciones}

\begin{enumerate}
    \item $$9x + 16y = 7$$
$$4y - 3x = 0$$

\item $$8x + 9y = 0$$
$$2x + 5y + 3y = \frac{7}{2}$$

\item $$\dfrac{x}{a+b}+\dfrac{y}{a+b}=\dfrac{1}{ab}$$
$$\dfrac{x}{b}+\dfrac{y}{a}=\dfrac{a^2+b^2}{a^2b^2}$$

\item $$2x + 3y + z = 1$$
$$6x - 2y - z = -14$$
$$3x + y - z = 1$$

\end{enumerate}


\section*{Resolver los siguientes problemas:}

\begin{enumerate}
    \item Seis veces el ancho de una sale excede en 4 m a la longitud de la sala, y si la longitud
aumentada en 3 mse divide entre el ancho, el cociente es 5 y el residuo es 3. Hallar las dimensiones de la sala.
    \item Si al doble de la edad de A se suma la edad B, se obtiene la edad de C aumentada en 32 años. Si al tercio de la edad de B se le suma el doble de la de C, se obtiene la de A aumentada en 9 años. Finalmente, el tercio de la suma de las edades de A y B es 1 año menos que la edad de C. Hallar las edades respectivas.
\end{enumerate}

\section*{Simplificar las siguientes expresiones:}

\begin{enumerate}
    \item $10^{3x-1}10^{4-x}$
    \item $\dfrac{3^x}{3^{1-x}}$
\end{enumerate}

\section*{Resolver las siguientes ecuaciones:}

\begin{enumerate}
    \item $25^{x+1} = 125^{2x}$
    \item $x^2e^x - 5xe^x=0$
    \item $ \log_x4 = \dfrac{2}{3}$
    \item $2\log_5x = \log5 (x^2 - 6x + 2) $
\end{enumerate}

\section*{Escribir las siguientes expresiones en términos de logaritmos más simples:}

\begin{enumerate}
    \item $\log_b\frac{m^5n^3}{\sqrt{p}}$
    \item $\log\sqrt[3]{x^2-y^2}$
\end{enumerate}

\section*{Escribir las siguientes expresiones en términos de un solo logaritmo con coeficiente 1:}

\begin{enumerate}
    \item $\frac{1}{3} \log_bw - 3 \log_b x - 5 \log_by$
    \item $5(\frac{1}{2}\log_bu - 2\log_bv)$
\end{enumerate}

\end{document}